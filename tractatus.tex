% Created 2022-03-15 Tue 06:32
% Intended LaTeX compiler: pdflatex
\documentclass[11pt]{article}
\usepackage[utf8]{inputenc}
\usepackage[T1]{fontenc}
\usepackage{graphicx}
\usepackage{longtable}
\usepackage{wrapfig}
\usepackage{rotating}
\usepackage[normalem]{ulem}
\usepackage{amsmath}
\usepackage{amssymb}
\usepackage{capt-of}
\usepackage{hyperref}
\author{Ludwig Wittgenstein}
\date{}
\title{Tractatus Logico-Philosophicus}
\hypersetup{
 pdfauthor={Ludwig Wittgenstein},
 pdftitle={Tractatus Logico-Philosophicus},
 pdfkeywords={},
 pdfsubject={},
 pdfcreator={Emacs 27.2 (Org mode 9.6)}, 
 pdflang={English}}
\begin{document}

\maketitle
\tableofcontents

Perhaps this book will be understood only by someone who has himself
already had the thoughts that are expressed in it--or at least similar
thoughts.--So it is not a textbook.--Its purpose would be achieved if it
gave pleasure to one person who read and understood it.

The book deals with the problems of philosophy, and shows, I believe, that
the reason why these problems are posed is that the logic of our language
is misunderstood. The whole sense of the book might be summed up the
following words: what can be said at all can be said clearly, and what we
cannot talk about we must pass over in silence.

Thus the aim of the book is to draw a limit to thought, or rather--not to
thought, but to the expression of thoughts: for in order to be able to draw
a limit to thought, we should have to find both sides of the limit
thinkable (i.e. we should have to be able to think what cannot be thought).

It will therefore only be in language that the limit can be drawn, and what
lies on the other side of the limit will simply be nonsense.

I do not wish to judge how far my efforts coincide with those of other
philosophers. Indeed, what I have written here makes no claim to novelty in
detail, and the reason why I give no sources is that it is a matter of
indifference to me whether the thoughts that I have had have been
anticipated by someone else.

I will only mention that I am indebted to Frege's great works and of the
writings of my friend Mr Bertrand Russell for much of the stimulation of my
thoughts.

If this work has any value, it consists in two things: the first is that
thoughts are expressed in it, and on this score the better the thoughts are
expressed--the more the nail has been hit on the head--the greater will be
its value.--Here I am conscious of having fallen a long way short of what
is possible. Simply because my powers are too slight for the accomplishment
of the task.--May others come and do it better.

On the other hand the truth of the thoughts that are here communicated
seems to me unassailable and definitive. I therefore believe myself to have
found, on all essential points, the final solution of the problems. And if
I am not mistaken in this belief, then the second thing in which the of
this work consists is that it shows how little is achieved when these
problem are solved.

L.W. Vienna, 1918

\section*{1}
\label{sec:org4901420}
The world is all that is the case.
\subsection*{1.1}
\label{sec:orgc31d3eb}
The world is the totality of facts, not of things.
\subsubsection*{1.11}
\label{sec:org905d8c4}
The world is determined by the facts, and by their being all the
facts.
\subsubsection*{1.12}
\label{sec:org1837cba}
For the totality of facts determines what is the case, and also
whatever is not the case.
\subsubsection*{1.13}
\label{sec:orgc4ec194}
The facts in logical space are the world.
\subsection*{1.2}
\label{sec:org320e903}
The world divides into facts.
\subsubsection*{1.21}
\label{sec:org2f3ccd4}
Each item can be the case or not the case while everything else
remains the same.
\section*{2}
\label{sec:org3453611}
What is the case--a fact--is the existence of states of affairs.
\subsubsection*{2.01}
\label{sec:org78c3de0}
A state of affairs (a state of things) is a combination of objects
(things).
\begin{itemize}
\item 2.011
\label{sec:org437acf1}
It is essential to things that they should be possible constituents
of states of affairs.
\item 2.012
\label{sec:org1ca9e49}
In logic nothing is accidental: if a thing can occur in a state of
affairs, the possibility of the state of affairs must be written into the
thing itself.
\begin{itemize}
\item 2.0121
\label{sec:org7bd7149}
It would seem to be a sort of accident, if it turned out that a
situation would fit a thing that could already exist entirely on its own.
If things can occur in states of affairs, this possibility must be in them
from the beginning. (Nothing in the province of logic can be merely
possible. Logic deals with every possibility and all possibilities are its
facts.) Just as we are quite unable to imagine spatial objects outside
space or temporal objects outside time, so too there is no object that we
can imagine excluded from the possibility of combining with others. If I
can imagine objects combined in states of affairs, I cannot imagine them
excluded from the possibility of such combinations.
\item 2.0122
\label{sec:org2867b5c}
Things are independent in so far as they can occur in all possible
situations, but this form of independence is a form of connexion with
states of affairs, a form of dependence. (It is impossible for words to
appear in two different roles: by themselves, and in propositions.)
\item 2.0123
\label{sec:org9655f9c}
If I know an object I also know all its possible occurrences in
states of affairs. (Every one of these possibilities must be part of the
nature of the object.) A new possibility cannot be discovered later.
\begin{itemize}
\item 2.01231
\label{sec:org68d6604}
If I am to know an object, thought I need not know its external
properties, I must know all its internal properties.
\end{itemize}
\item 2.0124
\label{sec:orgdc07af9}
If all objects are given, then at the same time all possible states
of affairs are also given.
\end{itemize}
\item 2.013
\label{sec:org1d95795}
Each thing is, as it were, in a space of possible states of affairs.
This space I can imagine empty, but I cannot imagine the thing without the
space.
\begin{itemize}
\item 2.0131
\label{sec:org191b741}
A spatial object must be situated in infinite space. (A spatial
point is an argument-place.) A speck in the visual field, thought it need
not be red, must have some colour: it is, so to speak, surrounded by colour-
space. Notes must have some pitch, objects of the sense of touch some
degree of hardness, and so on.
\end{itemize}
\item 2.014
\label{sec:orgda61591}
Objects contain the possibility of all situations.
\begin{itemize}
\item 2.0141
\label{sec:org8e1352a}
The possibility of its occurring in states of affairs is the form of
an object.
\end{itemize}
\end{itemize}
\subsubsection*{2.02}
\label{sec:org2de9ffa}
Objects are simple.
\begin{itemize}
\item 2.0201
\label{sec:orgce8bb94}
Every statement about complexes can be resolved into a statement
about their constituents and into the propositions that describe the
complexes completely.
\end{itemize}
\item 2.021
\label{sec:org0560956}
Objects make up the substance of the world. That is why they cannot
be composite.
\begin{itemize}
\item 2.0211
\label{sec:org7a7fbaa}
If they world had no substance, then whether a proposition had sense
would depend on whether another proposition was true.
\item 2.0212
\label{sec:org2ef7d73}
In that case we could not sketch any picture of the world (true or
false).
\end{itemize}
\item 2.022
\label{sec:org1de305a}
It is obvious that an imagined world, however difference it may be
from the real one, must have something-- a form--in common with it.
\item 2.023
\label{sec:org3cb3b7c}
Objects are just what constitute this unalterable form.
\begin{itemize}
\item 2.0231
\label{sec:orgd3adf0e}
The substance of the world can only determine a form, and not any
material properties. For it is only by means of propositions that material
properties are represented--only by the configuration of objects that they
are produced.
\item 2.0232
\label{sec:org0c1fa2e}
In a manner of speaking, objects are colourless.
\item 2.0233
\label{sec:org726607e}
If two objects have the same logical form, the only distinction
between them, apart from their external properties, is that they are
different.
\begin{itemize}
\item 2.02331
\label{sec:org81ecd28}
Either a thing has properties that nothing else has, in which case
we can immediately use a description to distinguish it from the others and
refer to it; or, on the other hand, there are several things that have the
whole set of their properties in common, in which case it is quite
impossible to indicate one of them. For it there is nothing to distinguish
a thing, I cannot distinguish it, since otherwise it would be distinguished
after all.
\end{itemize}
\end{itemize}
\item 2.024
\label{sec:orgf8e999e}
The substance is what subsists independently of what is the case.
\item 2.025
\label{sec:orgcea14ba}
It is form and content.
\begin{itemize}
\item 2.0251
\label{sec:orgec3bb6c}
Space, time, colour (being coloured) are forms of objects.
\end{itemize}
\item 2.026
\label{sec:org68f3274}
There must be objects, if the world is to have unalterable form.
\item 2.027
\label{sec:org2d63961}
Objects, the unalterable, and the subsistent are one and the same.
\begin{itemize}
\item 2.0271
\label{sec:org84196b5}
Objects are what is unalterable and subsistent; their configuration
is what is changing and unstable.
\item 2.0272
\label{sec:orgb6a634c}
The configuration of objects produces states of affairs.
\end{itemize}
\end{itemize}
\subsubsection*{2.03}
\label{sec:org0483bfa}
In a state of affairs objects fit into one another like the links of a
chain.
\begin{itemize}
\item 2.031
\label{sec:org8eaa830}
In a state of affairs objects stand in a determinate relation to one
another.
\item 2.032
\label{sec:org87ff97e}
The determinate way in which objects are connected in a state of
affairs is the structure of the state of affairs.
\item 2.033
\label{sec:org52a8be6}
Form is the possibility of structure.
\item 2.034
\label{sec:org8477258}
The structure of a fact consists of the structures of states of
affairs.
\end{itemize}
\subsubsection*{2.04}
\label{sec:orga469f79}
The totality of existing states of affairs is the world.
\subsubsection*{2.05}
\label{sec:orgc2012ce}
The totality of existing states of affairs also determines which
states of affairs do not exist.
\subsubsection*{2.06}
\label{sec:org824095b}
The existence and non-existence of states of affairs is reality. (We
call the existence of states of affairs a positive fact, and their non-
existence a negative fact.)
\begin{itemize}
\item 2.061
\label{sec:orga5a10ac}
States of affairs are independent of one another.
\item 2.062
\label{sec:org7cf0eb3}
From the existence or non-existence of one state of affairs it is
impossible to infer the existence or non-existence of another.
\item 2.063
\label{sec:org06e18c4}
The sum-total of reality is the world.
\end{itemize}
\subsection*{2.1}
\label{sec:orgd0fe7e1}
We picture facts to ourselves.
\subsubsection*{2.11}
\label{sec:org65d0e3b}
A picture presents a situation in logical space, the existence and non-
existence of states of affairs.
\subsubsection*{2.12}
\label{sec:org57081dc}
A picture is a model of reality.
\subsubsection*{2.13}
\label{sec:org4cfd864}
In a picture objects have the elements of the picture corresponding to
them.
\begin{itemize}
\item 2.131
\label{sec:org538df39}
In a picture the elements of the picture are the representatives of
objects.
\end{itemize}
\subsubsection*{2.14}
\label{sec:orga5e691e}
What constitutes a picture is that its elements are related to one
another in a determinate way.
\begin{itemize}
\item 2.141
\label{sec:orgbdb8d9b}
A picture is a fact.
\end{itemize}
\subsubsection*{2.15}
\label{sec:org2f085c0}
The fact that the elements of a picture are related to one another in
a determinate way represents that things are related to one another in the
same way. Let us call this connexion of its elements the structure of the
picture, and let us call the possibility of this structure the pictorial
form of the picture.
\begin{itemize}
\item 2.151
\label{sec:org308057b}
Pictorial form is the possibility that things are related to one
another in the same way as the elements of the picture.
\begin{itemize}
\item 2.1511
\label{sec:orgee4d88b}
That is how a picture is attached to reality; it reaches right out
to it.
\item 2.1512
\label{sec:org0defcf2}
It is laid against reality like a measure.
\begin{itemize}
\item 2.15121
\label{sec:org49d8d0c}
Only the end-points of the graduating lines actually touch the
object that is to be measured.
\end{itemize}
\item 2.1514
\label{sec:org37569c3}
So a picture, conceived in this way, also includes the pictorial
relationship, which makes it into a picture.
\item 2.1515
\label{sec:org3e7a926}
These correlations are, as it were, the feelers of the picture's
elements, with which the picture touches reality.
\end{itemize}
\end{itemize}
\subsubsection*{2.16}
\label{sec:org2b80dcf}
If a fact is to be a picture, it must have something in common with
what it depicts.
\begin{itemize}
\item 2.161
\label{sec:org95fdc92}
There must be something identical in a picture and what it depicts,
to enable the one to be a picture of the other at all.
\end{itemize}
\subsubsection*{2.17}
\label{sec:org83e371b}
What a picture must have in common with reality, in order to be able
to depict it--correctly or incorrectly--in the way that it does, is its
pictorial form.
\begin{itemize}
\item 2.171
\label{sec:org46263fc}
A picture can depict any reality whose form it has. A spatial picture
can depict anything spatial, a coloured one anything coloured, etc.
\item 2.172
\label{sec:org154d895}
A picture cannot, however, depict its pictorial form: it displays it.
\item 2.173
\label{sec:org569c727}
A picture represents its subject from a position outside it. (Its
standpoint is its representational form.) That is why a picture represents
its subject correctly or incorrectly.
\item 2.174
\label{sec:orgcb326bf}
A picture cannot, however, place itself outside its representational
form.
\end{itemize}
\subsubsection*{2.18}
\label{sec:org1540955}
What any picture, of whatever form, must have in common with reality,
in order to be able to depict it--correctly or incorrectly--in any way at
all, is logical form, i.e. the form of reality.
\begin{itemize}
\item 2.181
\label{sec:orgcb8b56d}
A picture whose pictorial form is logical form is called a logical
picture.
\item 2.182
\label{sec:org6caf870}
Every picture is at the same time a logical one. (On the other hand,
not every picture is, for example, a spatial one.)
\end{itemize}
\subsubsection*{2.19}
\label{sec:org83c74a4}
Logical pictures can depict the world.
\subsection*{2.2}
\label{sec:orgecfb403}
A picture has logico-pictorial form in common with what it depicts.
\begin{itemize}
\item 2.201
\label{sec:org9f74c18}
A picture depicts reality by representing a possibility of existence
and non-existence of states of affairs.
\item 2.202
\label{sec:org79eb5e2}
A picture contains the possibility of the situation that it
represents.
\item 2.203
\label{sec:orgf86a898}
A picture agrees with reality or fails to agree; it is correct or
incorrect, true or false.
\end{itemize}
\subsubsection*{2.22}
\label{sec:org85b9394}
What a picture represents it represents independently of its truth or
falsity, by means of its pictorial form.
\begin{itemize}
\item 2.221
\label{sec:org166ed71}
What a picture represents is its sense.
\item 2.222
\label{sec:orgc4e90a6}
The agreement or disagreement or its sense with reality constitutes
its truth or falsity.
\item 2.223
\label{sec:org54d199a}
In order to tell whether a picture is true or false we must compare
it with reality.
\item 2.224
\label{sec:orge6346b0}
It is impossible to tell from the picture alone whether it is true or
false.
\item 2.225
\label{sec:org73a768e}
There are no pictures that are true a priori.
\end{itemize}
\section*{3}
\label{sec:org69cab6a}
A logical picture of facts is a thought.
\begin{itemize}
\item 3.001
\label{sec:org24591d0}
'A state of affairs is thinkable': what this means is that we can
picture it to ourselves.
\end{itemize}
\subsubsection*{3.01}
\label{sec:org474da34}
The totality of true thoughts is a picture of the world.
\subsubsection*{3.02}
\label{sec:org25a18b4}
A thought contains the possibility of the situation of which it is the
thought. What is thinkable is possible too.
\subsubsection*{3.03}
\label{sec:orge3452e0}
Thought can never be of anything illogical, since, if it were, we
should have to think illogically.
\begin{itemize}
\item 3.031
\label{sec:org1e2f59c}
It used to be said that God could create anything except what would
be contrary to the laws of logic.The truth is that we could not say what an
'illogical' world would look like.
\item 3.032
\label{sec:orgb0d4cc7}
It is as impossible to represent in language anything that
'contradicts logic' as it is in geometry to represent by its coordinates a
figure that contradicts the laws of space, or to give the coordinates of a
point that does not exist.
\begin{itemize}
\item 3.0321
\label{sec:orgf70a9d1}
Though a state of affairs that would contravene the laws of physics
can be represented by us spatially, one that would contravene the laws of
geometry cannot.
\end{itemize}
\end{itemize}
\subsubsection*{3.04}
\label{sec:orgd37cfb2}
It a thought were correct a priori, it would be a thought whose
possibility ensured its truth.
\subsubsection*{3.05}
\label{sec:org9ffe1cf}
A priori knowledge that a thought was true would be possible only it
its truth were recognizable from the thought itself (without anything a to
compare it with).
\subsection*{3.1}
\label{sec:orgcdd538b}
In a proposition a thought finds an expression that can be perceived by
the senses.
\subsubsection*{3.11}
\label{sec:org740ec55}
We use the perceptible sign of a proposition (spoken or written, etc.)
as a projection of a possible situation. The method of projection is to
think of the sense of the proposition.
\subsubsection*{3.12}
\label{sec:org9768576}
I call the sign with which we express a thought a propositional
sign.And a proposition is a propositional sign in its projective relation
to the world.
\subsubsection*{3.13}
\label{sec:org224ea6c}
A proposition, therefore, does not actually contain its sense, but
does contain the possibility of expressing it. ('The content of a
proposition' means the content of a proposition that has sense.) A
proposition contains the form, but not the content, of its sense.
\subsubsection*{3.14}
\label{sec:org8898d8f}
What constitutes a propositional sign is that in its elements (the
words) stand in a determinate relation to one another. A propositional sign
is a fact.
\begin{itemize}
\item 3.141
\label{sec:org9b3e6d9}
A proposition is not a blend of words.(Just as a theme in music is
not a blend of notes.) A proposition is articulate.
\item 3.142
\label{sec:org1b2440c}
Only facts can express a sense, a set of names cannot.
\item 3.143
\label{sec:org5c6e845}
Although a propositional sign is a fact, this is obscured by the
usual form of expression in writing or print. For in a printed proposition,
for example, no essential difference is apparent between a propositional
sign and a word. (That is what made it possible for Frege to call a
proposition a composite name.)
\begin{itemize}
\item 3.1431
\label{sec:org6247bc7}
The essence of a propositional sign is very clearly seen if we
imagine one composed of spatial objects (such as tables, chairs, and books)
instead of written signs.
\item 3.1432
\label{sec:org86553b2}
Instead of, 'The complex sign ``aRb'' says that a stands to b in the
relation R' we ought to put, 'That ``a'' stands to ``b'' in a certain relation
says that aRb.'
\end{itemize}
\item 3.144
\label{sec:org2539195}
Situations can be described but not given names.
\end{itemize}
\subsection*{3.2}
\label{sec:orgb485885}
In a proposition a thought can be expressed in such a way that elements
of the propositional sign correspond to the objects of the thought.
\begin{itemize}
\item 3.201
\label{sec:org9088682}
I call such elements 'simple signs', and such a proposition 'complete
analysed'.
\item 3.202
\label{sec:orgff304d6}
The simple signs employed in propositions are called names.
\item 3.203
\label{sec:orgb415785}
A name means an object. The object is its meaning. ('A' is the same
sign as 'A'.)
\end{itemize}
\subsubsection*{3.21}
\label{sec:org70a15c0}
The configuration of objects in a situation corresponds to the
configuration of simple signs in the propositional sign.
\begin{itemize}
\item 3.221
\label{sec:orge92ccdc}
Objects can only be named. Signs are their representatives. I can
only speak about them: I cannot put them into words. Propositions can only
say how things are, not what they are.
\end{itemize}
\subsubsection*{3.23}
\label{sec:orgde1beb6}
The requirement that simple signs be possible is the requirement that
sense be determinate.
\subsubsection*{3.24}
\label{sec:orgbf120e6}
A proposition about a complex stands in an internal relation to a
proposition about a constituent of the complex. A complex can be given only
by its description, which will be right or wrong. A proposition that
mentions a complex will not be nonsensical, if the complex does not exits,
but simply false. When a propositional element signifies a complex, this
can be seen from an indeterminateness in the propositions in which it
occurs. In such cases we know that the proposition leaves something
undetermined. (In fact the notation for generality contains a prototype.)
The contraction of a symbol for a complex into a simple symbol can be
expressed in a definition.
\subsubsection*{3.25}
\label{sec:org53d2528}
A proposition cannot be dissected any further by means of a
definition: it is a primitive sign.
\begin{itemize}
\item 3.261
\label{sec:org2876b70}
Every sign that has a definition signifies via the signs that serve
to define it; and the definitions point the way. Two signs cannot signify
in the same manner if one is primitive and the other is defined by means of
primitive signs. Names cannot be anatomized by means of definitions. (Nor
can any sign that has a meaning independently and on its own.)
\item 3.262
\label{sec:org0dc7dc2}
What signs fail to express, their application shows. What signs slur
over, their application says clearly.
\item 3.263
\label{sec:orgea53731}
The meanings of primitive signs can be explained by means of
elucidations. Elucidations are propositions that stood if the meanings of
those signs are already known.
\end{itemize}
\subsection*{3.3}
\label{sec:orgd231657}
Only propositions have sense; only in the nexus of a proposition does a
name have meaning.
\subsubsection*{3.31}
\label{sec:orgf8e7cdf}
I call any part of a proposition that characterizes its sense an
expression (or a symbol). (A proposition is itself an expression.)
Everything essential to their sense that propositions can have in common
with one another is an expression. An expression is the mark of a form and
a content.
\begin{itemize}
\item 3.311
\label{sec:orgbed3a7f}
An expression presupposes the forms of all the propositions in which
it can occur. It is the common characteristic mark of a class of
propositions.
\item 3.312
\label{sec:org6add7e7}
It is therefore presented by means of the general form of the
propositions that it characterizes. In fact, in this form the expression
will be constant and everything else variable.
\item 3.313
\label{sec:org5c83d63}
Thus an expression is presented by means of a variable whose values
are the propositions that contain the expression. (In the limiting case the
variable becomes a constant, the expression becomes a proposition.) I call
such a variable a 'propositional variable'.
\item 3.314
\label{sec:org2f38a6e}
An expression has meaning only in a proposition. All variables can be
construed as propositional variables. (Even variable names.)
\item 3.315
\label{sec:org41b7d76}
If we turn a constituent of a proposition into a variable, there is a
class of propositions all of which are values of the resulting variable
proposition. In general, this class too will be dependent on the meaning
that our arbitrary conventions have given to parts of the original
proposition. But if all the signs in it that have arbitrarily determined
meanings are turned into variables, we shall still get a class of this
kind. This one, however, is not dependent on any convention, but solely on
the nature of the pro position. It corresponds to a logical form--a logical
prototype.
\item 3.316
\label{sec:org32b174c}
What values a propositional variable may take is something that is
stipulated. The stipulation of values is the variable.
\item 3.317
\label{sec:orgfa84d61}
To stipulate values for a propositional variable is to give the
propositions whose common characteristic the variable is. The stipulation
is a description of those propositions. The stipulation will therefore be
concerned only with symbols, not with their meaning. And the only thing
essential to the stipulation is that it is merely a description of symbols
and states nothing about what is signified. How the description of the
propositions is produced is not essential.
\item 3.318
\label{sec:org869f15f}
Like Frege and Russell I construe a proposition as a function of the
expressions contained in it.
\end{itemize}
\subsubsection*{3.32}
\label{sec:orgdf4e9d3}
A sign is what can be perceived of a symbol.
\begin{itemize}
\item 3.321
\label{sec:org00fe6d4}
So one and the same sign (written or spoken, etc.) can be common to
two different symbols--in which case they will signify in different ways.
\item 3.322
\label{sec:org6de505f}
Our use of the same sign to signify two different objects can never
indicate a common characteristic of the two, if we use it with two
different modes of signification. For the sign, of course, is arbitrary. So
we could choose two different signs instead, and then what would be left in
common on the signifying side?
\item 3.323
\label{sec:org9de3bcf}
In everyday language it very frequently happens that the same word
has different modes of signification--and so belongs to different symbols--
or that two words that have different modes of signification are employed
in propositions in what is superficially the same way. Thus the word 'is'
figures as the copula, as a sign for identity, and as an expression for
existence; 'exist' figures as an intransitive verb like 'go', and
'identical' as an adjective; we speak of something, but also of something's
happening. (In the proposition, 'Green is green'--where the first word is
the proper name of a person and the last an adjective--these words do not
merely have different meanings: they are different symbols.)
\item 3.324
\label{sec:org399719b}
In this way the most fundamental confusions are easily produced (the
whole of philosophy is full of them).
\item 3.325
\label{sec:org51385ac}
In order to avoid such errors we must make use of a sign-language
that excludes them by not using the same sign for different symbols and by
not using in a superficially similar way signs that have different modes of
signification: that is to say, a sign-language that is governed by logical
grammar--by logical syntax. (The conceptual notation of Frege and Russell
is such a language, though, it is true, it fails to exclude all mistakes.)
\item 3.326
\label{sec:orgd47c915}
In order to recognize a symbol by its sign we must observe how it is
used with a sense.
\item 3.327
\label{sec:org4814124}
A sign does not determine a logical form unless it is taken together
with its logico-syntactical employment.
\item 3.328
\label{sec:org874ee0f}
If a sign is useless, it is meaningless. That is the point of Occam's
maxim. (If everything behaves as if a sign had meaning, then it does have
meaning.)
\end{itemize}
\subsubsection*{3.33}
\label{sec:orgd2ca939}
In logical syntax the meaning of a sign should never play a role. It
must be possible to establish logical syntax without mentioning the meaning
of a sign: only the description of expressions may be presupposed.
\begin{itemize}
\item 3.331
\label{sec:org4b719d5}
From this observation we turn to Russell's 'theory of types'. It can
be seen that Russell must be wrong, because he had to mention the meaning
of signs when establishing the rules for them.
\item 3.332
\label{sec:orgd3caf7b}
No proposition can make a statement about itself, because a
propositional sign cannot be contained in itself (that is the whole of the
'theory of types').
\item 3.333
\label{sec:org80279d4}
The reason why a function cannot be its own argument is that the sign
for a function already contains the prototype of its argument, and it
cannot contain itself. For let us suppose that the function F(fx) could be
its own argument: in that case there would be a proposition 'F(F(fx))', in
which the outer function F and the inner function F must have different
meanings, since the inner one has the form O(f(x)) and the outer one has
the form Y(O(fx)). Only the letter 'F' is common to the two functions, but
the letter by itself signifies nothing. This immediately becomes clear if
instead of 'F(Fu)' we write '(do) : F(Ou) . Ou = Fu'. That disposes of
Russell's paradox.
\item 3.334
\label{sec:org7860a3a}
The rules of logical syntax must go without saying, once we know how
each individual sign signifies.
\end{itemize}
\subsubsection*{3.34}
\label{sec:org5cfe32e}
A proposition possesses essential and accidental features. Accidental
features are those that result from the particular way in which the
propositional sign is produced. Essential features are those without which
the proposition could not express its sense.
\begin{itemize}
\item 3.341
\label{sec:org754923d}
So what is essential in a proposition is what all propositions that
can express the same sense have in common. And similarly, in general, what
is essential in a symbol is what all symbols that can serve the same
purpose have in common.
\begin{itemize}
\item 3.3411
\label{sec:orge54bd51}
So one could say that the real name of an object was what all
symbols that signified it had in common. Thus, one by one, all kinds of
composition would prove to be unessential to a name.
\end{itemize}
\item 3.342
\label{sec:orgd62bbe8}
Although there is something arbitrary in our notations, this much is
not arbitrary--that when we have determined one thing arbitrarily,
something else is necessarily the case. (This derives from the essence of
notation.)
\begin{itemize}
\item 3.3421
\label{sec:org9da63c9}
A particular mode of signifying may be unimportant but it is always
important that it is a possible mode of signifying. And that is generally
so in philosophy: again and again the individual case turns out to be
unimportant, but the possibility of each individual case discloses
something about the essence of the world.
\end{itemize}
\item 3.343
\label{sec:org604223f}
Definitions are rules for translating from one language into another.
Any correct sign-language must be translatable into any other in accordance
with such rules: it is this that they all have in common.
\item 3.344
\label{sec:orgb72c1bb}
What signifies in a symbol is what is common to all the symbols that
the rules of logical syntax allow us to substitute for it.
\begin{itemize}
\item 3.3441
\label{sec:org211a366}
For instance, we can express what is common to all notations for
truth-functions in the following way: they have in common that, for
example, the notation that uses 'Pp' ('not p') and 'p C g' ('p or g') can
be substituted for any of them. (This serves to characterize the way in
which something general can be disclosed by the possibility of a specific
notation.)
\item 3.3442
\label{sec:org497f562}
Nor does analysis resolve the sign for a complex in an arbitrary
way, so that it would have a different resolution every time that it was
incorporated in a different proposition.
\end{itemize}
\end{itemize}
\subsection*{3.4}
\label{sec:orgdc41dba}
A proposition determines a place in logical space. The existence of
this logical place is guaranteed by the mere existence of the constituents--
by the existence of the proposition with a sense.
\subsubsection*{3.41}
\label{sec:orgce973e9}
The propositional sign with logical co-ordinates--that is the logical
place.
\begin{itemize}
\item 3.411
\label{sec:org66068a4}
In geometry and logic alike a place is a possibility: something can
exist in it.
\end{itemize}
\subsubsection*{3.42}
\label{sec:org9c01a7e}
A proposition can determine only one place in logical space:
nevertheless the whole of logical space must already be given by it.
(Otherwise negation, logical sum, logical product, etc., would introduce
more and more new elements in co-ordination.) (The logical scaffolding
surrounding a picture determines logical space. The force of a proposition
reaches through the whole of logical space.)
\subsection*{3.5}
\label{sec:orgd5db521}
A propositional sign, applied and thought out, is a thought.
\section*{4}
\label{sec:orgdafa17d}
A thought is a proposition with a sense.
\begin{itemize}
\item 4.001
\label{sec:orgdabfe81}
The totality of propositions is language.
\item 4.022
\label{sec:org8a83e95}
Man possesses the ability to construct languages capable of
expressing every sense, without having any idea how each word has meaning
or what its meaning is--just as people speak without knowing how the
individual sounds are produced. Everyday language is a part of the human
organism and is no less complicated than it. It is not humanly possible to
gather immediately from it what the logic of language is. Language
disguises thought. So much so, that from the outward form of the clothing
it is impossible to infer the form of the thought beneath it, because the
outward form of the clothing is not designed to reveal the form of the
body, but for entirely different purposes. The tacit conventions on which
the understanding of everyday language depends are enormously complicated.
\item 4.003
\label{sec:orgf3ea532}
Most of the propositions and questions to be found in philosophical
works are not false but nonsensical. Consequently we cannot give any answer
to questions of this kind, but can only point out that they are
nonsensical. Most of the propositions and questions of philosophers arise
from our failure to understand the logic of our language. (They belong to
the same class as the question whether the good is more or less identical
than the beautiful.) And it is not surprising that the deepest problems are
in fact not problems at all.
\begin{itemize}
\item 4.0031
\label{sec:org32a9fb1}
All philosophy is a 'critique of language' (though not in Mauthner's
sense). It was Russell who performed the service of showing that the
apparent logical form of a proposition need not be its real one.
\end{itemize}
\end{itemize}
\subsubsection*{4.01}
\label{sec:org72b6732}
A proposition is a picture of reality. A proposition is a model of
reality as we imagine it.
\begin{itemize}
\item 4.011
\label{sec:org2af48d4}
At first sight a proposition--one set out on the printed page, for
example--does not seem to be a picture of the reality with which it is
concerned. But neither do written notes seem at first sight to be a picture
of a piece of music, nor our phonetic notation (the alphabet) to be a
picture of our speech. And yet these sign-languages prove to be pictures,
even in the ordinary sense, of what they represent.
\item 4.012
\label{sec:org074ff75}
It is obvious that a proposition of the form 'aRb' strikes us as a
picture. In this case the sign is obviously a likeness of what is
signified.
\item 4.013
\label{sec:orga5df801}
And if we penetrate to the essence of this pictorial character, we
see that it is not impaired by apparent irregularities (such as the use
[sharp] of and [flat] in musical notation). For even these irregularities
depict what they are intended to express; only they do it in a different
way.
\item 4.014
\label{sec:org6fa75f1}
A gramophone record, the musical idea, the written notes, and the
sound-waves, all stand to one another in the same internal relation of
depicting that holds between language and the world. They are all
constructed according to a common logical pattern. (Like the two youths in
the fairy-tale, their two horses, and their lilies. They are all in a
certain sense one.)
\begin{itemize}
\item 4.0141
\label{sec:orgc19e1dd}
There is a general rule by means of which the musician can obtain
the symphony from the score, and which makes it possible to derive the
symphony from the groove on the gramophone record, and, using the first
rule, to derive the score again. That is what constitutes the inner
similarity between these things which seem to be constructed in such
entirely different ways. And that rule is the law of projection which
projects the symphony into the language of musical notation. It is the rule
for translating this language into the language of gramophone records.
\end{itemize}
\item 4.015
\label{sec:org9373a1a}
The possibility of all imagery, of all our pictorial modes of
expression, is contained in the logic of depiction.
\item 4.016
\label{sec:orgfbd74a5}
In order to understand the essential nature of a proposition, we
should consider hieroglyphic script, which depicts the facts that it
describes. And alphabetic script developed out of it without losing what
was essential to depiction.
\end{itemize}
\subsubsection*{4.02}
\label{sec:orga308db4}
We can see this from the fact that we understand the sense of a
propositional sign without its having been explained to us.
\begin{itemize}
\item 4.021
\label{sec:orgbef4fd1}
A proposition is a picture of reality: for if I understand a
proposition, I know the situation that it represents. And I understand the
proposition without having had its sense explained to me.
\item 4.022
\label{sec:org895f6ea}
A proposition shows its sense. A proposition shows how things stand
if it is true. And it says that they do so stand.
\item 4.023
\label{sec:org9967506}
A proposition must restrict reality to two alternatives: yes or no.
In order to do that, it must describe reality completely. A proposition is
a description of a state of affairs. Just as a description of an object
describes it by giving its external properties, so a proposition describes
reality by its internal properties. A proposition constructs a world with
the help of a logical scaffolding, so that one can actually see from the
proposition how everything stands logically if it is true. One can draw
inferences from a false proposition.
\item 4.024
\label{sec:org91e7d42}
To understand a proposition means to know what is the case if it is
true. (One can understand it, therefore, without knowing whether it is
true.) It is understood by anyone who understands its constituents.
\item 4.025
\label{sec:org7ffad81}
When translating one language into another, we do not proceed by
translating each proposition of the one into a proposition of the other,
but merely by translating the constituents of propositions. (And the
dictionary translates not only substantives, but also verbs, adjectives,
and conjunctions, etc.; and it treats them all in the same way.)
\item 4.026
\label{sec:org81b9346}
The meanings of simple signs (words) must be explained to us if we
are to understand them. With propositions, however, we make ourselves
understood.
\item 4.027
\label{sec:orgde833f0}
It belongs to the essence of a proposition that it should be able to
communicate a new sense to us.
\end{itemize}
\subsubsection*{4.03}
\label{sec:orgbc88676}
A proposition must use old expressions to communicate a new sense. A
proposition communicates a situation to us, and so it must be essentially
connected with the situation. And the connexion is precisely that it is its
logical picture. A proposition states something only in so far as it is a
picture.
\begin{itemize}
\item 4.031
\label{sec:orgcda6efc}
In a proposition a situation is, as it were, constructed by way of
experiment. Instead of, 'This proposition has such and such a sense, we can
simply say, 'This proposition represents such and such a situation'.
\begin{itemize}
\item 4.0311
\label{sec:orgc091685}
One name stands for one thing, another for another thing, and they
are combined with one another. In this way the whole group--like a tableau
vivant--presents a state of affairs.
\item 4.0312
\label{sec:orgc594c9a}
The possibility of propositions is based on the principle that
objects have signs as their representatives. My fundamental idea is that
the 'logical constants' are not representatives; that there can be no
representatives of the logic of facts.
\end{itemize}
\item 4.032
\label{sec:orgebd2b61}
It is only in so far as a proposition is logically articulated that
it is a picture of a situation. (Even the proposition, 'Ambulo', is
composite: for its stem with a different ending yields a different sense,
and so does its ending with a different stem.)
\end{itemize}
\subsubsection*{4.04}
\label{sec:org927354b}
In a proposition there must be exactly as many distinguishable parts
as in the situation that it represents. The two must possess the same
logical (mathematical) multiplicity. (Compare Hertz's Mechanics on
dynamical models.)
\begin{itemize}
\item 4.041
\label{sec:orgc333543}
This mathematical multiplicity, of course, cannot itself be the
subject of depiction. One cannot get away from it when depicting.
\begin{itemize}
\item 4.0411
\label{sec:org262e262}
If, for example, we wanted to express what we now write as '(x) .
fx' by putting an affix in front of 'fx'--for instance by writing 'Gen. fx'-
-it would not be adequate: we should not know what was being generalized.
If we wanted to signalize it with an affix 'g'--for instance by writing
'f(xg)'--that would not be adequate either: we should not know the scope of
the generality-sign. If we were to try to do it by introducing a mark into
the argument-places--for instance by writing '(G,G) . F(G,G)' --it would
not be adequate: we should not be able to establish the identity of the
variables. And so on. All these modes of signifying are inadequate because
they lack the necessary mathematical multiplicity.
\item 4.0412
\label{sec:orgddd83fe}
For the same reason the idealist's appeal to 'spatial spectacles' is
inadequate to explain the seeing of spatial relations, because it cannot
explain the multiplicity of these relations.
\end{itemize}
\end{itemize}
\subsubsection*{4.05}
\label{sec:org4daa67b}
Reality is compared with propositions.
\subsubsection*{4.06}
\label{sec:org5fbd581}
A proposition can be true or false only in virtue of being a picture
of reality.
\begin{itemize}
\item 4.061
\label{sec:orgeaeec7a}
It must not be overlooked that a proposition has a sense that is
independent of the facts: otherwise one can easily suppose that true and
false are relations of equal status between signs and what they signify. In
that case one could say, for example, that 'p' signified in the true way
what 'Pp' signified in the false way, etc.
\item 4.062
\label{sec:orge721a74}
Can we not make ourselves understood with false propositions just as
we have done up till now with true ones?--So long as it is known that they
are meant to be false.--No! For a proposition is true if we use it to say
that things stand in a certain way, and they do; and if by 'p' we mean Pp
and things stand as we mean that they do, then, construed in the new way,
'p' is true and not false.
\begin{itemize}
\item 4.0621
\label{sec:org75009cc}
But it is important that the signs 'p' and 'Pp' can say the same
thing. For it shows that nothing in reality corresponds to the sign 'P'.
The occurrence of negation in a proposition is not enough to characterize
its sense (PPp = p). The propositions 'p' and 'Pp' have opposite sense, but
there corresponds to them one and the same reality.
\end{itemize}
\item 4.063
\label{sec:org0226ec7}
An analogy to illustrate the concept of truth: imagine a black spot
on white paper: you can describe the shape of the spot by saying, for each
point on the sheet, whether it is black or white. To the fact that a point
is black there corresponds a positive fact, and to the fact that a point is
white (not black), a negative fact. If I designate a point on the sheet (a
truth-value according to Frege), then this corresponds to the supposition
that is put forward for judgement, etc. etc. But in order to be able to say
that a point is black or white, I must first know when a point is called
black, and when white: in order to be able to say,'``p'' is true (or false)',
I must have determined in what circumstances I call 'p' true, and in so
doing I determine the sense of the proposition. Now the point where the
simile breaks down is this: we can indicate a point on the paper even if we
do not know what black and white are, but if a proposition has no sense,
nothing corresponds to it, since it does not designatea thing (a truth-
value) which might have properties called 'false' or 'true'. The verb of a
proposition is not 'is true' or 'is false', as Frege thought: rather, that
which 'is true' must already contain the verb.
\item 4.064
\label{sec:org2016ca9}
Every proposition must already have a sense: it cannot be given a
sense by affirmation. Indeed its sense is just what is affirmed. And the
same applies to negation, etc.
\begin{itemize}
\item 4.0641
\label{sec:org81e8df2}
One could say that negation must be related to the logical place
determined by the negated proposition. The negating proposition determines
a logical place different from that of the negated proposition. The
negating proposition determines a logical place with the help of the
logical place of the negated proposition. For it describes it as lying
outside the latter's logical place. The negated proposition can be negated
again, and this in itself shows that what is negated is already a
proposition, and not merely something that is prelimary to a proposition.
\end{itemize}
\end{itemize}
\subsection*{4.1}
\label{sec:org319515b}
Propositions represent the existence and non-existence of states of
affairs.
\subsubsection*{4.11}
\label{sec:org966eee5}
The totality of true propositions is the whole of natural science (or
the whole corpus of the natural sciences).
\begin{itemize}
\item 4.111
\label{sec:orga5cf7a8}
Philosophy is not one of the natural sciences. (The word 'philosophy'
must mean something whose place is above or below the natural sciences, not
beside them.)
\item 4.112
\label{sec:org7dc96cf}
Philosophy aims at the logical clarification of thoughts. Philosophy
is not a body of doctrine but an activity. A philosophical work consists
essentially of elucidations. Philosophy does not result in 'philosophical
propositions', but rather in the clarification of propositions. Without
philosophy thoughts are, as it were, cloudy and indistinct: its task is to
make them clear and to give them sharp boundaries.
\begin{itemize}
\item 4.1121
\label{sec:org3cdf469}
Psychology is no more closely related to philosophy than any other
natural science. Theory of knowledge is the philosophy of psychology. Does
not my study of sign-language correspond to the study of thought-processes,
which philosophers used to consider so essential to the philosophy of
logic? Only in most cases they got entangled in unessential psychological
investigations, and with my method too there is an analogous risk.
\item 4.1122
\label{sec:org6cebdd9}
Darwin's theory has no more to do with philosophy than any other
hypothesis in natural science.
\end{itemize}
\item 4.113
\label{sec:orgbd9372a}
Philosophy sets limits to the much disputed sphere of natural
science.
\item 4.114
\label{sec:orgabbdcf8}
It must set limits to what can be thought; and, in doing so, to what
cannot be thought. It must set limits to what cannot be thought by working
outwards through what can be thought.
\item 4.115
\label{sec:org76a5eb5}
It will signify what cannot be said, by presenting clearly what can
be said.
\item 4.116
\label{sec:org93e8392}
Everything that can be thought at all can be thought clearly.
Everything that can be put into words can be put clearly. 4.12 Propositions
can represent the whole of reality, but they cannot represent what they
must have in common with reality in order to be able to represent it--
logical form. In order to be able to represent logical form, we should have
to be able to station ourselves with propositions somewhere outside logic,
that is to say outside the world.
\item 4.121
\label{sec:org0e3b0bf}
Propositions cannot represent logical form: it is mirrored in them.
What finds its reflection in language, language cannot represent. What
expresses itself in language, we cannot express by means of language.
Propositions show the logical form of reality. They display it.
\begin{itemize}
\item 4.1211
\label{sec:org3f50d60}
Thus one proposition 'fa' shows that the object a occurs in its
sense, two propositions 'fa' and 'ga' show that the same object is
mentioned in both of them. If two propositions contradict one another, then
their structure shows it; the same is true if one of them follows from the
other. And so on.
\item 4.1212
\label{sec:org33f2d80}
What can be shown, cannot be said.
\item 4.1213
\label{sec:orgcd89e85}
Now, too, we understand our feeling that once we have a sign-
language in which everything is all right, we already have a correct
logical point of view.
\end{itemize}
\item 4.122
\label{sec:orged474fc}
In a certain sense we can talk about formal properties of objects and
states of affairs, or, in the case of facts, about structural properties:
and in the same sense about formal relations and structural relations.
(Instead of 'structural property' I also say 'internal property'; instead
of 'structural relation', 'internal relation'. I introduce these
expressions in order to indicate the source of the confusion between
internal relations and relations proper (external relations), which is very
widespread among philosophers.) It is impossible, however, to assert by
means of propositions that such internal properties and relations obtain:
rather, this makes itself manifest in the propositions that represent the
relevant states of affairs and are concerned with the relevant objects.
\begin{itemize}
\item 4.1221
\label{sec:org06b376e}
An internal property of a fact can also be bed a feature of that
fact (in the sense in which we speak of facial features, for example).
\end{itemize}
\item 4.123
\label{sec:org4098d01}
A property is internal if it is unthinkable that its object should
not possess it. (This shade of blue and that one stand, eo ipso, in the
internal relation of lighter to darker. It is unthinkable that these two
objects should not stand in this relation.) (Here the shifting use of the
word 'object' corresponds to the shifting use of the words 'property' and
'relation'.)
\item 4.124
\label{sec:org9e7110e}
The existence of an internal property of a possible situation is not
expressed by means of a proposition: rather, it expresses itself in the
proposition representing the situation, by means of an internal property of
that proposition. It would be just as nonsensical to assert that a
proposition had a formal property as to deny it.
\begin{itemize}
\item 4.1241
\label{sec:org0f7f6aa}
It is impossible to distinguish forms from one another by saying
that one has this property and another that property: for this presupposes
that it makes sense to ascribe either property to either form.
\end{itemize}
\item 4.125
\label{sec:org5128ddf}
The existence of an internal relation between possible situations
expresses itself in language by means of an internal relation between the
propositions representing them.
\begin{itemize}
\item 4.1251
\label{sec:orge257550}
Here we have the answer to the vexed question 'whether all relations
are internal or external'.
\item 4.1252
\label{sec:org4a0c737}
I call a series that is ordered by an internal relation a series of
forms. The order of the number-series is not governed by an external
relation but by an internal relation. The same is true of the series of
propositions 'aRb', '(d : c) : aRx . xRb', '(d x,y) : aRx . xRy . yRb', and
so forth. (If b stands in one of these relations to a, I call b a successor
of a.)
\end{itemize}
\item 4.126
\label{sec:org3689b6f}
We can now talk about formal concepts, in the same sense that we
speak of formal properties. (I introduce this expression in order to
exhibit the source of the confusion between formal concepts and concepts
proper, which pervades the whole of traditional logic.) When something
falls under a formal concept as one of its objects, this cannot be
expressed by means of a proposition. Instead it is shown in the very sign
for this object. (A name shows that it signifies an object, a sign for a
number that it signifies a number, etc.) Formal concepts cannot, in fact,
be represented by means of a function, as concepts proper can. For their
characteristics, formal properties, are not expressed by means of
functions. The expression for a formal property is a feature of certain
symbols. So the sign for the characteristics of a formal concept is a
distinctive feature of all symbols whose meanings fall under the concept.
So the expression for a formal concept is a propositional variable in which
this distinctive f
eature alone is constant.
\item 4.127
\label{sec:org2ba3028}
The propositional variable signifies the formal concept, and its
values signify the objects that fall under the concept.
\begin{itemize}
\item 4.1271
\label{sec:orge4338dd}
Every variable is the sign for a formal concept. For every variable
represents a constant form that all its values possess, and this can be
regarded as a formal property of those values.
\item 4.1272
\label{sec:org636775f}
Thus the variable name 'x' is the proper sign for the pseudo-concept
object. Wherever the word 'object' ('thing', etc.) is correctly used, it is
expressed in conceptual notation by a variable name. For example, in the
proposition, 'There are 2 objects which. . .', it is expressed by ' (dx,y)
\ldots{} '. Wherever it is used in a different way, that is as a proper concept-
word, nonsensical pseudo-propositions are the result. So one cannot say,
for example, 'There are objects', as one might say, 'There are books'. And
it is just as impossible to say, 'There are 100 objects', or, 'There are !0
objects'. And it is nonsensical to speak of the total number of objects.
The same applies to the words 'complex', 'fact', 'function', 'number', etc.
They all signify formal concepts, and are represented in conceptual
notation by variables, not by functions or classes (as Frege and Russell
believed). '1 is a number', 'There is only one zero', and all similar
expressions are nonsensical. (It is just as nonsensical to say, 'There is
only one 1', as it would be to say, '2 + 2 at 3 o'clock equals 4'.)
\begin{itemize}
\item 4.12721
\label{sec:org3823265}
A formal concept is given immediately any object falling under it
is given. It is not possible, therefore, to introduce as primitive ideas
objects belonging to a formal concept and the formal concept itself. So it
is impossible, for example, to introduce as primitive ideas both the
concept of a function and specific functions, as Russell does; or the
concept of a number and particular numbers.
\end{itemize}
\item 4.1273
\label{sec:org278079d}
If we want to express in conceptual notation the general
proposition, 'b is a successor of a', then we require an expression for the
general term of the series of forms 'aRb', '(d : c) : aRx . xRb', '(d x,y)
\begin{verbatim}
aRx . xRy . yRb', ... , In order to express the general term of a series
\end{verbatim}

of forms, we must use a variable, because the concept 'term of that series
of forms' is a formal concept. (This is what Frege and Russell overlooked:
consequently the way in which they want to express general propositions
like the one above is incorrect; it contains a vicious circle.) We can
determine the general term of a series of forms by giving its first term
and the general form of the operation that produces the next term out of
the proposition that precedes it.
\item 4.1274
\label{sec:org3e80367}
To ask whether a formal concept exists is nonsensical. For no
proposition can be the answer to such a question. (So, for example, the
question, 'Are there unanalysable subject-predicate propositions?' cannot
be asked.)
\end{itemize}
\item 4.128
\label{sec:orgb1ac537}
Logical forms are without number. Hence there are no preeminent
numbers in logic, and hence there is no possibility of philosophical monism
or dualism, etc.
\end{itemize}
\subsection*{4.2}
\label{sec:orgdb7fae3}
The sense of a proposition is its agreement and disagreement with
possibilities of existence and non-existence of states of affairs. 4.21 The
simplest kind of proposition, an elementary proposition, asserts the
existence of a state of affairs.
\begin{itemize}
\item 4.211
\label{sec:orgbc9fa96}
It is a sign of a proposition's being elementary that there can be no
elementary proposition contradicting it.
\end{itemize}
\subsubsection*{4.22}
\label{sec:org4c91fcc}
An elementary proposition consists of names. It is a nexus, a
concatenation, of names.
\begin{itemize}
\item 4.221
\label{sec:org17f520a}
It is obvious that the analysis of propositions must bring us to
elementary propositions which consist of names in immediate combination.
This raises the question how such combination into propositions comes
about.
\begin{itemize}
\item 4.2211
\label{sec:orgb5cba1f}
Even if the world is infinitely complex, so that every fact consists
of infinitely many states of affairs and every state of affairs is composed
of infinitely many objects, there would still have to be objects and states
of affairs.
\end{itemize}
\end{itemize}
\subsubsection*{4.23}
\label{sec:org7f66eed}
It is only in the nexus of an elementary proposition that a name
occurs in a proposition.
\subsubsection*{4.24}
\label{sec:org33edf41}
Names are the simple symbols: I indicate them by single letters ('x',
'y', 'z'). I write elementary propositions as functions of names, so that
they have the form 'fx', 'O (x,y)', etc. Or I indicate them by the letters
'p', 'q', 'r'.
\begin{itemize}
\item 4.241
\label{sec:org0f3b9f8}
When I use two signs with one and the same meaning, I express this by
putting the sign '=' between them. So 'a = b' means that the sign 'b' can
be substituted for the sign 'a'. (If I use an equation to introduce a new
sign 'b', laying down that it shall serve as a substitute for a sign a that
is already known, then, like Russell, I write the equation-- definition--in
the form 'a = b Def.' A definition is a rule dealing with signs.)
\item 4.242
\label{sec:orgdf35666}
Expressions of the form 'a = b' are, therefore, mere representational
devices. They state nothing about the meaning of the signs 'a' and 'b'.
\item 4.243
\label{sec:org46fd6d7}
Can we understand two names without knowing whether they signify the
same thing or two different things?--Can we understand a proposition in
which two names occur without knowing whether their meaning is the same or
different? Suppose I know the meaning of an English word and of a German
word that means the same: then it is impossible for me to be unaware that
they do mean the same; I must be capable of translating each into the
other. Expressions like 'a = a', and those derived from them, are neither
elementary propositions nor is there any other way in which they have
sense. (This will become evident later.)
\end{itemize}
\subsubsection*{4.25}
\label{sec:org504e43d}
If an elementary proposition is true, the state of affairs exists: if
an elementary proposition is false, the state of affairs does not exist.
\subsubsection*{4.26}
\label{sec:org19d5867}
If all true elementary propositions are given, the result is a
complete description of the world. The world is completely described by
giving all elementary propositions, and adding which of them are true and
which false. For n states of affairs, there are possibilities of existence
and non-existence. Of these states of affairs any combination can exist and
the remainder not exist.
\subsubsection*{4.28}
\label{sec:orge023029}
There correspond to these combinations the same number of
possibilities of truth--and falsity--for n elementary propositions.
\subsection*{4.3}
\label{sec:org9a3ba33}
Truth-possibilities of elementary propositions mean Possibilities of
existence and non-existence of states of affairs.
\subsubsection*{4.31}
\label{sec:org4963e67}
We can represent truth-possibilities by schemata of the following kind
('T' means 'true', 'F' means 'false'; the rows of 'T's' and 'F's' under the
row of elementary propositions symbolize their truth-possibilities in a way
that can easily be understood):
\subsection*{4.4}
\label{sec:org96dbace}
A proposition is an expression of agreement and disagreement with truth-
possibilities of elementary propositions.
\subsubsection*{4.41}
\label{sec:org75f16f3}
Truth-possibilities of elementary propositions are the conditions of
the truth and falsity of propositions.
\begin{itemize}
\item 4.411
\label{sec:org65a4aa6}
It immediately strikes one as probable that the introduction of
elementary propositions provides the basis for understanding all other
kinds of proposition. Indeed the understanding of general propositions
palpably depends on the understanding of elementary propositions.
\end{itemize}
\subsubsection*{4.42}
\label{sec:org3083ed4}
For n elementary propositions there are ways in which a proposition
can agree and disagree with their truth possibilities.
\subsubsection*{4.43}
\label{sec:org2ea98d3}
We can express agreement with truth-possibilities by correlating the
mark 'T' (true) with them in the schema. The absence of this mark means
disagreement.
\begin{itemize}
\item 4.431
\label{sec:orgb5ca169}
The expression of agreement and disagreement with the truth
possibilities of elementary propositions expresses the truth-conditions of
a proposition. A proposition is the expression of its truth-conditions.
(Thus Frege was quite right to use them as a starting point when he
explained the signs of his conceptual notation. But the explanation of the
concept of truth that Frege gives is mistaken: if 'the true' and 'the
false' were really objects, and were the arguments in Pp etc., then Frege's
method of determining the sense of 'Pp' would leave it absolutely
undetermined.)
\end{itemize}
\subsubsection*{4.44}
\label{sec:orgcc7878b}
The sign that results from correlating the mark 'I`` with truth-
possibilities is a propositional sign.
\begin{itemize}
\item 4.441
\label{sec:orgab9ad49}
It is clear that a complex of the signs 'F' and 'T' has no object (or
complex of objects) corresponding to it, just as there is none
corresponding to the horizontal and vertical lines or to the brackets.--
There are no 'logical objects'. Of course the same applies to all signs
that express what the schemata of 'T's' and 'F's' express.
\item 4.442
\label{sec:org6d82501}
For example, the following is a propositional sign: (Frege's
'judgement stroke' '|-' is logically quite meaningless: in the works of
Frege (and Russell) it simply indicates that these authors hold the
propositions marked with this sign to be true. Thus '|-' is no more a
component part of a proposition than is, for instance, the proposition's
number. It is quite impossible for a proposition to state that it itself is
true.) If the order or the truth-possibilities in a scheme is fixed once
and for all by a combinatory rule, then the last column by itself will be
an expression of the truth-conditions. If we now write this column as a
row, the propositional sign will become '(TT-T) (p,q)' or more explicitly
'(TTFT) (p,q)' (The number of places in the left-hand pair of brackets is
determined by the number of terms in the right-hand pair.)
\end{itemize}
\subsubsection*{4.45}
\label{sec:org0d7327a}
For n elementary propositions there are Ln possible groups of truth-
conditions. The groups of truth-conditions that are obtainable from the
truth-possibilities of a given number of elementary propositions can be
arranged in a series.
\subsubsection*{4.46}
\label{sec:orgb56f876}
Among the possible groups of truth-conditions there are two extreme
cases. In one of these cases the proposition is true for all the truth-
possibilities of the elementary propositions. We say that the truth-
conditions are tautological. In the second case the proposition is false
for all the truth-possibilities: the truth-conditions are contradictory .
In the first case we call the proposition a tautology; in the second, a
contradiction.
\begin{itemize}
\item 4.461
\label{sec:org7ed19b2}
Propositions show what they say; tautologies and contradictions show
that they say nothing. A tautology has no truth-conditions, since it is
unconditionally true: and a contradiction is true on no condition.
Tautologies and contradictions lack sense. (Like a point from which two
arrows go out in opposite directions to one another.) (For example, I know
nothing about the weather when I know that it is either raining or not
raining.)
\begin{itemize}
\item 4.46211
\label{sec:orgd490cac}
Tautologies and contradictions are not, however, nonsensical. They
are part of the symbolism, much as '0' is part of the symbolism of
arithmetic.
\end{itemize}
\item 4.462
\label{sec:orgaf74776}
Tautologies and contradictions are not pictures of reality. They do
not represent any possible situations. For the former admit all possible
situations, and latter none . In a tautology the conditions of agreement
with the world--the representational relations--cancel one another, so that
it does not stand in any representational relation to reality.
\item 4.463
\label{sec:org2792f56}
The truth-conditions of a proposition determine the range that it
leaves open to the facts. (A proposition, a picture, or a model is, in the
negative sense, like a solid body that restricts the freedom of movement of
others, and in the positive sense, like a space bounded by solid substance
in which there is room for a body.) A tautology leaves open to reality the
whole--the infinite whole--of logical space: a contradiction fills the
whole of logical space leaving no point of it for reality. Thus neither of
them can determine reality in any way.
\item 4.464
\label{sec:org8487757}
A tautology's truth is certain, a proposition's possible, a
contradiction's impossible. (Certain, possible, impossible: here we have
the first indication of the scale that we need in the theory of
probability.)
\item 4.465
\label{sec:org9ee5e97}
The logical product of a tautology and a proposition says the same
thing as the proposition. This product, therefore, is identical with the
proposition. For it is impossible to alter what is essential to a symbol
without altering its sense.
\item 4.466
\label{sec:orge539726}
What corresponds to a determinate logical combination of signs is a
determinate logical combination of their meanings. It is only to the
uncombined signs that absolutely any combination corresponds. In other
words, propositions that are true for every situation cannot be
combinations of signs at all, since, if they were, only determinate
combinations of objects could correspond to them. (And what is not a
logical combination has no combination of objects corresponding to it.)
Tautology and contradiction are the limiting cases--indeed the
disintegration--of the combination of signs.
\begin{itemize}
\item 4.4661
\label{sec:org44c6022}
Admittedly the signs are still combined with one another even in
tautologies and contradictions--i.e. they stand in certain relations to one
another: but these relations have no meaning, they are not essential to the
symbol .
\end{itemize}
\end{itemize}
\subsection*{4.5}
\label{sec:org98629f8}
It now seems possible to give the most general propositional form: that
is, to give a description of the propositions of any sign-language
whatsoever in such a way that every possible sense can be expressed by a
symbol satisfying the description, and every symbol satisfying the
description can express a sense, provided that the meanings of the names
are suitably chosen. It is clear that only what is essential to the most
general propositional form may be included in its description--for
otherwise it would not be the most general form. The existence of a general
propositional form is proved by the fact that there cannot be a proposition
whose form could not have been foreseen (i.e. constructed). The general
form of a proposition is: This is how things stand.
\subsubsection*{4.51}
\label{sec:org9edc601}
Suppose that I am given all elementary propositions: then I can simply
ask what propositions I can construct out of them. And there I have all
propositions, and that fixes their limits.
\subsubsection*{4.52}
\label{sec:orga19241c}
Propositions comprise all that follows from the totality of all
elementary propositions (and, of course, from its being the totality of
them all ). (Thus, in a certain sense, it could be said that all
propositions were generalizations of elementary propositions.)
\subsubsection*{4.53}
\label{sec:org4273229}
The general propositional form is a variable.
\section*{5}
\label{sec:org7a2ad09}
A proposition is a truth-function of elementary propositions. (An
elementary proposition is a truth-function of itself.)
\subsubsection*{5.01}
\label{sec:org25a1bd4}
Elementary propositions are the truth-arguments of propositions.
\subsubsection*{5.02}
\label{sec:org42863b1}
The arguments of functions are readily confused with the affixes of
names. For both arguments and affixes enable me to recognize the meaning of
the signs containing them. For example, when Russell writes '+c', the 'c'
is an affix which indicates that the sign as a whole is the addition-sign
for cardinal numbers. But the use of this sign is the result of arbitrary
convention and it would be quite possible to choose a simple sign instead
of '+c'; in 'Pp' however, 'p' is not an affix but an argument: the sense of
'Pp' cannot be understood unless the sense of 'p' has been understood
already. (In the name Julius Caesar 'Julius' is an affix. An affix is
always part of a description of the object to whose name we attach it: e.g.
the Caesar of the Julian gens.) If I am not mistaken, Frege's theory about
the meaning of propositions and functions is based on the confusion between
an argument and an affix. Frege regarded the propositions of logic as
names, and their arguments as the affixes of those names.
\subsection*{5.1}
\label{sec:orgcf2fddc}
Truth-functions can be arranged in series. That is the foundation of
the theory of probability.
\begin{itemize}
\item 5.101
\label{sec:orgc9f80b6}
The truth-functions of a given number of elementary propositions can
always be set out in a schema of the following kind: (TTTT) (p, q)
Tautology (If p then p, and if q then q.) (p z p . q z q) (FTTT) (p, q) In
words : Not both p and q. (P(p . q)) (TFTT) (p, q) `` : If q then p. (q z p)
(TTFT) (p, q) '' : If p then q. (p z q) (TTTF) (p, q) `` : p or q. (p C q)
(FFTT) (p, q) '' : Not g. (Pq) (FTFT) (p, q) `` : Not p. (Pp) (FTTF) (p, q) ''
\begin{verbatim}
p or q, but not both. (p . Pq : C : q . Pp) (TFFT) (p, q) " : If p then
\end{verbatim}

p, and if q then p. (p + q) (TFTF) (p, q) `` : p (TTFF) (p, q) '' : q (FFFT)
(p, q) `` : Neither p nor q. (Pp . Pq or p | q) (FFTF) (p, q) '' : p and not
\begin{enumerate}
\item (p . Pq) (FTFF) (p, q) `` : q and not p. (q . Pp) (TFFF) (p,q) '' : q and
\item (q . p) (FFFF) (p, q) Contradiction (p and not p, and q and not q.) (p .
\end{enumerate}
Pp . q . Pq) I will give the name truth-grounds of a proposition to those
truth-possibilities of its truth-arguments that make it true.
\end{itemize}
\subsubsection*{5.11}
\label{sec:org29217b1}
If all the truth-grounds that are common to a number of propositions
are at the same time truth-grounds of a certain proposition, then we say
that the truth of that proposition follows from the truth of the others.
\subsubsection*{5.12}
\label{sec:orge4d84a5}
In particular, the truth of a proposition 'p' follows from the truth
of another proposition 'q' is all the truth-grounds of the latter are truth-
grounds of the former.
\begin{itemize}
\item 5.121
\label{sec:orge64f5e4}
The truth-grounds of the one are contained in those of the other: p
follows from q.
\item 5.122
\label{sec:orgc739d7e}
If p follows from q, the sense of 'p' is contained in the sense of
'q'.
\item 5.123
\label{sec:org8628c22}
If a god creates a world in which certain propositions are true, then
by that very act he also creates a world in which all the propositions that
follow from them come true. And similarly he could not create a world in
which the proposition 'p' was true without creating all its objects.
\item 5.124
\label{sec:org78165c5}
A proposition affirms every proposition that follows from it.
\begin{itemize}
\item 5.1241
\label{sec:org04d81bf}
'p . q' is one of the propositions that affirm 'p' and at the same
time one of the propositions that affirm 'q'. Two propositions are opposed
to one another if there is no proposition with a sense, that affirms them
both. Every proposition that contradicts another negate it.
\end{itemize}
\end{itemize}
\subsubsection*{5.13}
\label{sec:orge58b87b}
When the truth of one proposition follows from the truth of others, we
can see this from the structure of the proposition.
\begin{itemize}
\item 5.131
\label{sec:org7b0a711}
If the truth of one proposition follows from the truth of others,
this finds expression in relations in which the forms of the propositions
stand to one another: nor is it necessary for us to set up these relations
between them, by combining them with one another in a single proposition;
on the contrary, the relations are internal, and their existence is an
immediate result of the existence of the propositions.
\begin{itemize}
\item 5.1311
\label{sec:orgf3fb296}
When we infer q from p C q and Pp, the relation between the
propositional forms of 'p C q' and 'Pp' is masked, in this case, by our
mode of signifying. But if instead of 'p C q' we write, for example, 'p|q .
\begin{center}
\begin{tabular}{llll}
. p & q', and instead of 'Pp', 'p & p' (p & q = neither p nor q), then the\\
\end{tabular}
\end{center}
inner connexion becomes obvious. (The possibility of inference from (x) .
fx to fa shows that the symbol (x) . fx itself has generality in it.)
\end{itemize}
\item 5.132
\label{sec:orgd5494aa}
If p follows from q, I can make an inference from q to p, deduce p
from q. The nature of the inference can be gathered only from the two
propositions. They themselves are the only possible justification of the
inference. 'Laws of inference', which are supposed to justify inferences,
as in the works of Frege and Russell, have no sense, and would be
superfluous.
\item 5.133
\label{sec:org516bfaf}
All deductions are made a priori.
\item 5.134
\label{sec:orgf348749}
One elementary proposition cannot be deduced form another.
\item 5.135
\label{sec:orgb8cd1bd}
There is no possible way of making an inference form the existence of
one situation to the existence of another, entirely different situation.
\item 5.136
\label{sec:orge4053f0}
There is no causal nexus to justify such an inference.
\begin{itemize}
\item 5.1361
\label{sec:org5001de5}
We cannot infer the events of the future from those of the present.
Belief in the causal nexus is superstition.
\item 5.1362
\label{sec:orgc847c67}
The freedom of the will consists in the impossibility of knowing
actions that still lie in the future. We could know them only if causality
were an inner necessity like that of logical inference.--The connexion
between knowledge and what is known is that of logical necessity. ('A knows
that p is the case', has no sense if p is a tautology.)
\item 5.1363
\label{sec:org63ef55e}
If the truth of a proposition does not follow from the fact that it
is self-evident to us, then its self-evidence in no way justifies our
belief in its truth.
\end{itemize}
\end{itemize}
\subsubsection*{5.14}
\label{sec:org0cb73c9}
If one proposition follows from another, then the latter says more
than the former, and the former less than the latter.
\begin{itemize}
\item 5.141
\label{sec:orga807fba}
If p follows from q and q from p, then they are one and same
proposition.
\item 5.142
\label{sec:orgd750432}
A tautology follows from all propositions: it says nothing.
\item 5.143
\label{sec:orgfc926f4}
Contradiction is that common factor of propositions which no
proposition has in common with another. Tautology is the common factor of
all propositions that have nothing in common with one another.
Contradiction, one might say, vanishes outside all propositions: tautology
vanishes inside them. Contradiction is the outer limit of propositions:
tautology is the unsubstantial point at their centre.
\end{itemize}
\subsubsection*{5.15}
\label{sec:orgcb768d9}
If Tr is the number of the truth-grounds of a proposition 'r', and if
Trs is the number of the truth-grounds of a proposition 's' that are at the
same time truth-grounds of 'r', then we call the ratio Trs : Tr the degree
of probability that the proposition 'r' gives to the proposition 's'. 5.151
In a schema like the one above in
5.101, let Tr be the number of 'T's' in the proposition r, and let Trs, be
the number of 'T's' in the proposition s that stand in columns in which the
proposition r has 'T's'. Then the proposition r gives to the proposition s
the probability Trs : Tr.
\begin{itemize}
\item 5.1511
\label{sec:orgb1fd7f6}
There is no special object peculiar to probability propositions.
\end{itemize}
\item 5.152
\label{sec:org0ffe9bd}
When propositions have no truth-arguments in common with one another,
we call them independent of one another. Two elementary propositions give
one another the probability 1/2. If p follows from q, then the proposition
'q' gives to the proposition 'p' the probability 1. The certainty of
logical inference is a limiting case of probability. (Application of this
to tautology and contradiction.)
\item 5.153
\label{sec:org519b539}
In itself, a proposition is neither probable nor improbable. Either
an event occurs or it does not: there is no middle way.
\item 5.154
\label{sec:orgc5f6b09}
Suppose that an urn contains black and white balls in equal numbers
(and none of any other kind). I draw one ball after another, putting them
back into the urn. By this experiment I can establish that the number of
black balls drawn and the number of white balls drawn approximate to one
another as the draw continues. So this is not a mathematical truth. Now, if
I say, 'The probability of my drawing a white ball is equal to the
probability of my drawing a black one', this means that all the
circumstances that I know of (including the laws of nature assumed as
hypotheses) give no more probability to the occurrence of the one event
than to that of the other. That is to say, they give each the probability
\end{itemize}
\subsection*{1/2}
\label{sec:org7a232b8}
as can easily be gathered from the above definitions. What I confirm by
the experiment is that the occurrence of the two events is independent of
the circumstances of which I have no more detailed knowledge.
\begin{itemize}
\item 5.155
\label{sec:orgc223ff8}
The minimal unit for a probability proposition is this: The
circumstances--of which I have no further knowledge--give such and such a
degree of probability to the occurrence of a particular event.
\item 5.156
\label{sec:org28cbe9e}
It is in this way that probability is a generalization. It involves a
general description of a propositional form. We use probability only in
default of certainty--if our knowledge of a fact is not indeed complete,
but we do know something about its form. (A proposition may well be an
incomplete picture of a certain situation, but it is always a complete
picture of something .) A probability proposition is a sort of excerpt from
other propositions.
\end{itemize}
\subsection*{5.2}
\label{sec:orgf5f0a58}
The structures of propositions stand in internal relations to one
another.
\subsubsection*{5.21}
\label{sec:orgdb8d0e3}
In order to give prominence to these internal relations we can adopt
the following mode of expression: we can represent a proposition as the
result of an operation that produces it out of other propositions (which
are the bases of the operation).
\subsubsection*{5.22}
\label{sec:org83721fa}
An operation is the expression of a relation between the structures of
its result and of its bases.
\subsubsection*{5.23}
\label{sec:org08e5453}
The operation is what has to be done to the one proposition in order
to make the other out of it.
\begin{itemize}
\item 5.231
\label{sec:org251081b}
And that will, of course, depend on their formal properties, on the
internal similarity of their forms.
\item 5.232
\label{sec:org88d37ed}
The internal relation by which a series is ordered is equivalent to
the operation that produces one term from another.
\item 5.233
\label{sec:orgd171f30}
Operations cannot make their appearance before the point at which one
proposition is generated out of another in a logically meaningful way; i.e.
the point at which the logical construction of propositions begins.
\item 5.234
\label{sec:org752e578}
Truth-functions of elementary propositions are results of operations
with elementary propositions as bases. (These operations I call truth-
operations.)
\begin{itemize}
\item 5.2341
\label{sec:orgcee65ba}
The sense of a truth-function of p is a function of the sense of p.
Negation, logical addition, logical multiplication, etc. etc. are
operations. (Negation reverses the sense of a proposition.)
\end{itemize}
\end{itemize}
\subsubsection*{5.24}
\label{sec:orgad1fc76}
An operation manifests itself in a variable; it shows how we can get
from one form of proposition to another. It gives expression to the
difference between the forms. (And what the bases of an operation and its
result have in common is just the bases themselves.)
\begin{itemize}
\item 5.241
\label{sec:org8c3e816}
An operation is not the mark of a form, but only of a difference
between forms.
\item 5.242
\label{sec:orgd550519}
The operation that produces 'q' from 'p' also produces 'r' from 'q',
and so on. There is only one way of expressing this: 'p', 'q', 'r', etc.
have to be variables that give expression in a general way to certain
formal relations.
\end{itemize}
\subsubsection*{5.25}
\label{sec:org09dd1fb}
The occurrence of an operation does not characterize the sense of a
proposition. Indeed, no statement is made by an operation, but only by its
result, and this depends on the bases of the operation. (Operations and
functions must not be confused with each other.)
\begin{itemize}
\item 5.251
\label{sec:org5df33ee}
A function cannot be its own argument, whereas an operation can take
one of its own results as its base.
\item 5.252
\label{sec:orgaaec182}
It is only in this way that the step from one term of a series of
forms to another is possible (from one type to another in the hierarchies
of Russell and Whitehead). (Russell and Whitehead did not admit the
possibility of such steps, but repeatedly availed themselves of it.)
\begin{itemize}
\item 5.2521
\label{sec:org4a961ed}
If an operation is applied repeatedly to its own results, I speak of
successive applications of it. ('O'O'O'a' is the result of three successive
applications of the operation 'O'E' to 'a'.) In a similar sense I speak of
successive applications of more than one operation to a number of
propositions.
\item 5.2522
\label{sec:org7d7db36}
Accordingly I use the sign '[a, x, O'x]' for the general term of the
series of forms a, O'a, O'O'a, \ldots{} . This bracketed expression is a
variable: the first term of the bracketed expression is the beginning of
the series of forms, the second is the form of a term x arbitrarily
selected from the series, and the third is the form of the term that
immediately follows x in the series.
\item 5.2523
\label{sec:org8fdba79}
The concept of successive applications of an operation is equivalent
to the concept 'and so on'.
\end{itemize}
\item 5.253
\label{sec:org8de0657}
One operation can counteract the effect of another. Operations can
cancel one another.
\item 5.254
\label{sec:orgae133ca}
An operation can vanish (e.g. negation in 'PPp' : PPp = p).
\end{itemize}
\subsection*{5.3}
\label{sec:orgf6019d9}
All propositions are results of truth-operations on elementary
propositions. A truth-operation is the way in which a truth-function is
produced out of elementary propositions. It is of the essence of truth-
operations that, just as elementary propositions yield a truth-function of
themselves, so too in the same way truth-functions yield a further truth-
function. When a truth-operation is applied to truth-functions of
elementary propositions, it always generates another truth-function of
elementary propositions, another proposition. When a truth-operation is
applied to the results of truth-operations on elementary propositions,
there is always a single operation on elementary propositions that has the
same result. Every proposition is the result of truth-operations on
elementary propositions.
\subsubsection*{5.31}
\label{sec:orgb51a2ec}
The schemata in 4.31 have a meaning even when 'p', 'q', 'r', etc. are
not elementary propositions. And it is easy to see that the propositional
sign in 4.442 expresses a single truth-function of elementary propositions
even when 'p' and 'q' are truth-functions of elementary propositions.
\subsubsection*{5.32}
\label{sec:orgaa56c47}
All truth-functions are results of successive applications to
elementary propositions of a finite number of truth-operations.
\subsection*{5.4}
\label{sec:org83eed2e}
At this point it becomes manifest that there are no 'logical objects'
or 'logical constants' (in Frege's and Russell's sense).
\subsubsection*{5.41}
\label{sec:org5abd192}
The reason is that the results of truth-operations on truth-functions
are always identical whenever they are one and the same truth-function of
elementary propositions.
\subsubsection*{5.42}
\label{sec:orgea0404e}
It is self-evident that C, z, etc. are not relations in the sense in
which right and left etc. are relations. The interdefinability of Frege's
and Russell's 'primitive signs' of logic is enough to show that they are
not primitive signs, still less signs for relations. And it is obvious that
the 'z' defined by means of 'P' and 'C' is identical with the one that
figures with 'P' in the definition of 'C'; and that the second 'C' is
identical with the first one; and so on.
\subsubsection*{5.43}
\label{sec:org617a7fe}
Even at first sight it seems scarcely credible that there should
follow from one fact p infinitely many others , namely PPp, PPPPp, etc. And
it is no less remarkable that the infinite number of propositions of logic
(mathematics) follow from half a dozen 'primitive propositions'. But in
fact all the propositions of logic say the same thing, to wit nothing.
\subsubsection*{5.44}
\label{sec:org2857fe7}
Truth-functions are not material functions. For example, an
affirmation can be produced by double negation: in such a case does it
follow that in some sense negation is contained in affirmation? Does 'PPp'
negate Pp, or does it affirm p--or both? The proposition 'PPp' is not about
negation, as if negation were an object: on the other hand, the possibility
of negation is already written into affirmation. And if there were an
object called 'P', it would follow that 'PPp' said something different from
what 'p' said, just because the one proposition would then be about P and
the other would not.
\begin{itemize}
\item 5.441
\label{sec:org0dce1ed}
This vanishing of the apparent logical constants also occurs in the
case of 'P(dx) . Pfx', which says the same as '(x) . fx', and in the case
of '(dx) . fx . x = a', which says the same as 'fa'.
\item 5.442
\label{sec:org7f14565}
If we are given a proposition, then with it we are also given the
results of all truth-operations that have it as their base.
\end{itemize}
\subsubsection*{5.45}
\label{sec:orgd90f076}
If there are primitive logical signs, then any logic that fails to
show clearly how they are placed relatively to one another and to justify
their existence will be incorrect. The construction of logic out of its
primitive signs must be made clear.
\begin{itemize}
\item 5.451
\label{sec:orga695f42}
If logic has primitive ideas, they must be independent of one
another. If a primitive idea has been introduced, it must have been
introduced in all the combinations in which it ever occurs. It cannot,
therefore, be introduced first for one combination and later reintroduced
for another. For example, once negation has been introduced, we must
understand it both in propositions of the form 'Pp' and in propositions
like 'P(p C q)', '(dx) . Pfx', etc. We must not introduce it first for the
one class of cases and then for the other, since it would then be left in
doubt whether its meaning were the same in both cases, and no reason would
have been given for combining the signs in the same way in both cases. (In
short, Frege's remarks about introducing signs by means of definitions (in
The Fundamental Laws of Arithmetic ) also apply, mutatis mutandis, to the
introduction of primitive signs.)
\item 5.452
\label{sec:orgf51610b}
The introduction of any new device into the symbolism of logic is
necessarily a momentous event. In logic a new device should not be
introduced in brackets or in a footnote with what one might call a
completely innocent air. (Thus in Russell and Whitehead's Principia
Mathematica there occur definitions and primitive propositions expressed in
words. Why this sudden appearance of words? It would require a
justification, but none is given, or could be given, since the procedure is
in fact illicit.) But if the introduction of a new device has proved
necessary at a certain point, we must immediately ask ourselves, 'At what
points is the employment of this device now unavoidable ?' and its place in
logic must be made clear.
\item 5.453
\label{sec:orgd3ad9d4}
All numbers in logic stand in need of justification. Or rather, it
must become evident that there are no numbers in logic. There are no pre-
eminent numbers.
\item 5.454
\label{sec:orgcbca3b8}
In logic there is no co-ordinate status, and there can be no
classification. In logic there can be no distinction between the general
and the specific.
\begin{itemize}
\item 5.4541
\label{sec:org703ad8b}
The solutions of the problems of logic must be simple, since they
set the standard of simplicity. Men have always had a presentiment that
there must be a realm in which the answers to questions are symmetrically
combined--a priori--to form a self-contained system. A realm subject to the
law: Simplex sigillum veri.
\end{itemize}
\end{itemize}
\subsubsection*{5.46}
\label{sec:orgb0995b1}
If we introduced logical signs properly, then we should also have
introduced at the same time the sense of all combinations of them; i.e. not
only 'p C q' but 'P(p C q)' as well, etc. etc. We should also have
introduced at the same time the effect of all possible combinations of
brackets. And thus it would have been made clear that the real general
primitive signs are not ' p C q', '(dx) . fx', etc. but the most general
form of their combinations.
\begin{itemize}
\item 5.461
\label{sec:orgb54fce8}
Though it seems unimportant, it is in fact significant that the
pseudo-relations of logic, such as C and z, need brackets--unlike real
relations. Indeed, the use of brackets with these apparently primitive
signs is itself an indication that they are not primitive signs. And surely
no one is going to believe brackets have an independent meaning. 5.4611
Signs for logical operations are punctuation-marks,
\end{itemize}
\subsubsection*{5.47}
\label{sec:org684bc07}
It is clear that whatever we can say in advance about the form of all
propositions, we must be able to say all at once . An elementary
proposition really contains all logical operations in itself. For 'fa' says
the same thing as '(dx) . fx . x = a' Wherever there is compositeness,
argument and function are present, and where these are present, we already
have all the logical constants. One could say that the sole logical
constant was what all propositions, by their very nature, had in common
with one another. But that is the general propositional form.
\begin{itemize}
\item 5.471
\label{sec:org28d6989}
The general propositional form is the essence of a proposition.
\begin{itemize}
\item 5.4711
\label{sec:org1a17a2d}
To give the essence of a proposition means to give the essence of
all description, and thus the essence of the world.
\end{itemize}
\item 5.472
\label{sec:org019b61a}
The description of the most general propositional form is the
description of the one and only general primitive sign in logic.
\item 5.473
\label{sec:org7bc28e0}
Logic must look after itself. If a sign is possible , then it is also
capable of signifying. Whatever is possible in logic is also permitted.
(The reason why 'Socrates is identical' means nothing is that there is no
property called 'identical'. The proposition is nonsensical because we have
failed to make an arbitrary determination, and not because the symbol, in
itself, would be illegitimate.) In a certain sense, we cannot make mistakes
in logic.
\begin{itemize}
\item 5.4731
\label{sec:org6447050}
Self-evidence, which Russell talked about so much, can become
dispensable in logic, only because language itself prevents every logical
mistake.--What makes logic a priori is the impossibility of illogical
thought.
\item 5.4732
\label{sec:org04d7c74}
We cannot give a sign the wrong sense.
\begin{itemize}
\item 5,47321
\label{sec:orgf9c3c1d}
Occam's maxim is, of course, not an arbitrary rule, nor one that is
justified by its success in practice: its point is that unnecessary units
in a sign-language mean nothing. Signs that serve one purpose are logically
equivalent, and signs that serve none are logically meaningless.
\end{itemize}
\item 5.4733
\label{sec:org3ec176a}
Frege says that any legitimately constructed proposition must have a
sense. And I say that any possible proposition is legitimately constructed,
and, if it has no sense, that can only be because we have failed to give a
meaning to some of its constituents. (Even if we think that we have done
so.) Thus the reason why 'Socrates is identical' says nothing is that we
have not given any adjectival meaning to the word 'identical'. For when it
appears as a sign for identity, it symbolizes in an entirely different way--
the signifying relation is a different one--therefore the symbols also are
entirely different in the two cases: the two symbols have only the sign in
common, and that is an accident.
\end{itemize}
\item 5.474
\label{sec:org65b3ddf}
The number of fundamental operations that are necessary depends
solely on our notation.
\item 5.475
\label{sec:org3c7bc62}
All that is required is that we should construct a system of signs
with a particular number of dimensions--with a particular mathematical
multiplicity
\item 5.476
\label{sec:org9d4dd21}
It is clear that this is not a question of a number of primitive
ideas that have to be signified, but rather of the expression of a rule.
\end{itemize}
\subsection*{5.5}
\label{sec:orgdcc220f}
Every truth-function is a result of successive applications to
elementary propositions of the operation '(-----T)(E, \ldots{}.)'. This
operation negates all the propositions in the right-hand pair of brackets,
and I call it the negation of those propositions.
\begin{itemize}
\item 5.501
\label{sec:orgfec9a0b}
When a bracketed expression has propositions as its terms--and the
order of the terms inside the brackets is indifferent--then I indicate it
by a sign of the form '(E)'. '(E)' is a variable whose values are terms of
the bracketed expression and the bar over the variable indicates that it is
the representative of ali its values in the brackets. (E.g. if E has the
three values P,Q, R, then (E) = (P, Q, R). ) What the values of the
variable are is something that is stipulated. The stipulation is a
description of the propositions that have the variable as their
representative. How the description of the terms of the bracketed
expression is produced is not essential. We can distinguish three kinds of
description: 1.Direct enumeration, in which case we can simply substitute
for the variable the constants that are its values; 2. giving a function fx
whose values for all values of x are the propositions to be described; 3.
giving a formal law that governs the construction of the propositions, in
which case the bracketed expression has as its members all the terms of a
series of forms.
\item 5.502
\label{sec:org6be9688}
So instead of '(-----T)(E, \ldots{}.)', I write 'N(E)'. N(E) is the
negation of all the values of the propositional variable E.
\item 5.503
\label{sec:org8ea7f39}
It is obvious that we can easily express how propositions may be
constructed with this operation, and how they may not be constructed with
it; so it must be possible to find an exact expression for this.
\end{itemize}
\subsubsection*{5.51}
\label{sec:orgf011931}
If E has only one value, then N(E) = Pp (not p); if it has two values,
then N(E) = Pp . Pq. (neither p nor g).
\begin{itemize}
\item 5.511
\label{sec:orga535ee2}
How can logic--all-embracing logic, which mirrors the world--use such
peculiar crotchets and contrivances? Only because they are all connected
with one another in an infinitely fine network, the great mirror.
\item 5.512
\label{sec:orga5a504f}
'Pp' is true if 'p' is false. Therefore, in the proposition 'Pp',
when it is true, 'p' is a false proposition. How then can the stroke 'P'
make it agree with reality? But in 'Pp' it is not 'P' that negates, it is
rather what is common to all the signs of this notation that negate p. That
is to say the common rule that governs the construction of 'Pp', 'PPPp',
'Pp C Pp', 'Pp . Pp', etc. etc. (ad inf.). And this common factor mirrors
negation.
\item 5.513
\label{sec:org211b873}
We might say that what is common to all symbols that affirm both p
and q is the proposition 'p . q'; and that what is common to all symbols
that affirm either p or q is the proposition 'p C q'. And similarly we can
say that two propositions are opposed to one another if they have nothing
in common with one another, and that every proposition has only one
negative, since there is only one proposition that lies completely outside
it. Thus in Russell's notation too it is manifest that 'q : p C Pp' says
the same thing as 'q', that 'p C Pq' says nothing.
\item 5.514
\label{sec:org82897ac}
Once a notation has been established, there will be in it a rule
governing the construction of all propositions that negate p, a rule
governing the construction of all propositions that affirm p, and a rule
governing the construction of all propositions that affirm p or q; and so
on. These rules are equivalent to the symbols; and in them their sense is
mirrored.
\item 5.515
\label{sec:org6a9afae}
It must be manifest in our symbols that it can only be propositions
that are combined with one another by 'C', '.', etc. And this is indeed the
case, since the symbol in 'p' and 'q' itself presupposes 'C', 'P', etc. If
the sign 'p' in 'p C q' does not stand for a complex sign, then it cannot
have sense by itself: but in that case the signs 'p C p', 'p . p', etc.,
which have the same sense as p, must also lack sense. But if 'p C p' has no
sense, then 'p C q' cannot have a sense either.
\begin{itemize}
\item 5.5151
\label{sec:org7bd3253}
Must the sign of a negative proposition be constructed with that of
the positive proposition? Why should it not be possible to express a
negative proposition by means of a negative fact? (E.g. suppose that ``a'
does not stand in a certain relation to `b'; then this might be used to say
that aRb was not the case.) But really even in this case the negative
proposition is constructed by an indirect use of the positive. The positive
proposition necessarily presupposes the existence of the negative
proposition and vice versa.
\end{itemize}
\end{itemize}
\subsubsection*{5.52}
\label{sec:org862e045}
If E has as its values all the values of a function fx for all values
of x, then N(E) = P(dx) . fx.
\begin{itemize}
\item 5.521
\label{sec:org03a1242}
I dissociate the concept all from truth-functions. Frege and Russell
introduced generality in association with logical productor logical sum.
This made it difficult to understand the propositions '(dx) . fx' and '(x)
. fx', in which both ideas are embedded.
\item 5.522
\label{sec:org9639366}
What is peculiar to the generality-sign is first, that it indicates a
logical prototype, and secondly, that it gives prominence to constants.
\item 5.523
\label{sec:orge4bca9b}
The generality-sign occurs as an argument.
\item 5.524
\label{sec:org1f41238}
If objects are given, then at the same time we are given all objects.
If elementary propositions are given, then at the same time all elementary
propositions are given.
\item 5.525
\label{sec:orgce27a29}
It is incorrect to render the proposition '(dx) . fx' in the words,
'fx is possible ' as Russell does. The certainty, possibility, or
impossibility of a situation is not expressed by a proposition, but by an
expression's being a tautology, a proposition with a sense, or a
contradiction. The precedent to which we are constantly inclined to appeal
must reside in the symbol itself.
\item 5.526
\label{sec:org520b77e}
We can describe the world completely by means of fully generalized
propositions, i.e. without first correlating any name with a particular
object.
\begin{itemize}
\item 5.5261
\label{sec:org22502e0}
A fully generalized proposition, like every other proposition, is
composite. (This is shown by the fact that in '(dx, O) . Ox' we have to
mention 'O' and 's' separately. They both, independently, stand in
signifying relations to the world, just as is the case in ungeneralized
propositions.) It is a mark of a composite symbol that it has something in
common with other symbols.
\item 5.5262
\label{sec:orgc8c84bc}
The truth or falsity of every proposition does make some alteration
in the general construction of the world. And the range that the totality
of elementary propositions leaves open for its construction is exactly the
same as that which is delimited by entirely general propositions. (If an
elementary proposition is true, that means, at any rate, one more true
elementary proposition.)
\end{itemize}
\end{itemize}
\subsubsection*{5.53}
\label{sec:org6cc3c69}
Identity of object I express by identity of sign, and not by using a
sign for identity. Difference of objects I express by difference of signs.
\begin{itemize}
\item 5.5301
\label{sec:orgbd5166c}
It is self-evident that identity is not a relation between objects.
This becomes very clear if one considers, for example, the proposition '(x)
\begin{verbatim}
fx . z . x = a'. What this proposition says is simply that only a
\end{verbatim}

satisfies the function f, and not that only things that have a certain
relation to a satisfy the function, Of course, it might then be said that
only a did have this relation to a; but in order to express that, we should
need the identity-sign itself.
\item 5.5302
\label{sec:org147bcad}
Russell's definition of '=' is inadequate, because according to it
we cannot say that two objects have all their properties in common. (Even
if this proposition is never correct, it still has sense .)
\item 5.5303
\label{sec:org18bb089}
Roughly speaking, to say of two things that they are identical is
nonsense, and to say of one thing that it is identical with itself is to
say nothing at all.
\end{itemize}
\item 5.531
\label{sec:org00036c3}
Thus I do not write 'f(a, b) . a = b', but 'f(a, a)' (or 'f(b, b));
and not 'f(a,b) . Pa = b', but 'f(a, b)'.
\item 5.532
\label{sec:orgdfb6bb4}
And analogously I do not write '(dx, y) . f(x, y) . x = y', but '(dx)
. f(x, x)'; and not '(dx, y) . f(x, y) . Px = y', but '(dx, y) . f(x, y)'.
\begin{itemize}
\item 5.5321
\label{sec:org0678b80}
Thus, for example, instead of '(x) : fx z x = a' we write '(dx) . fx
. z : (dx, y) . fx. fy'. And the proposition, 'Only one x satisfies f( )',
will read '(dx) . fx : P(dx, y) . fx . fy'.
\end{itemize}
\item 5.533
\label{sec:org997392f}
The identity-sign, therefore, is not an essential constituent of
conceptual notation.
\item 5.534
\label{sec:org4f48ef0}
And now we see that in a correct conceptual notation pseudo-
propositions like 'a = a', 'a = b . b = c . z a = c', '(x) . x = x', '(dx)
. x = a', etc. cannot even be written down.
\item 5.535
\label{sec:org4a497c6}
This also disposes of all the problems that were connected with such
pseudo-propositions. All the problems that Russell's 'axiom of infinity'
brings with it can be solved at this point. What the axiom of infinity is
intended to say would express itself in language through the existence of
infinitely many names with different meanings.
\begin{itemize}
\item 5.5351
\label{sec:orgb4226d5}
There are certain cases in which one is tempted to use expressions
of the form 'a = a' or 'p z p' and the like. In fact, this happens when one
wants to talk about prototypes, e.g. about proposition, thing, etc. Thus in
Russell's Principles of Mathematics 'p is a proposition'--which is nonsense-
-was given the symbolic rendering 'p z p' and placed as an hypothesis in
front of certain propositions in order to exclude from their argument-
places everything but propositions. (It is nonsense to place the hypothesis
'p z p' in front of a proposition, in order to ensure that its arguments
shall have the right form, if only because with a non-proposition as
argument the hypothesis becomes not false but nonsensical, and because
arguments of the wrong kind make the proposition itself nonsensical, so
that it preserves itself from wrong arguments just as well, or as badly, as
the hypothesis without sense that was appended for that purpose.)
\item 5.5352
\label{sec:orgf64eac0}
In the same way people have wanted to express, 'There are no things
', by writing 'P(dx) . x = x'. But even if this were a proposition, would
it not be equally true if in fact 'there were things' but they were not
identical with themselves?
\end{itemize}
\end{itemize}
\subsubsection*{5.54}
\label{sec:org41433d0}
In the general propositional form propositions occur in other
propositions only as bases of truth-operations.
\begin{itemize}
\item 5.541
\label{sec:org656643c}
At first sight it looks as if it were also possible for one
proposition to occur in another in a different way. Particularly with
certain forms of proposition in psychology, such as 'A believes that p is
the case' and A has the thought p', etc. For if these are considered
superficially, it looks as if the proposition p stood in some kind of
relation to an object A. (And in modern theory of knowledge (Russell,
Moore, etc.) these propositions have actually been construed in this way.)
\item 5.542
\label{sec:org424f25e}
It is clear, however, that 'A believes that p', 'A has the thought
p', and 'A says p' are of the form '``p'' says p': and this does not involve
a correlation of a fact with an object, but rather the correlation of facts
by means of the correlation of their objects.
\begin{itemize}
\item 5.5421
\label{sec:orgaf0061b}
This shows too that there is no such thing as the soul--the subject,
etc.--as it is conceived in the superficial psychology of the present day.
Indeed a composite soul would no longer be a soul.
\item 5.5422
\label{sec:org1a333cb}
The correct explanation of the form of the proposition, 'A makes the
judgement p', must show that it is impossible for a judgement to be a piece
of nonsense. (Russell's theory does not satisfy this requirement.)
\item 5.5423
\label{sec:org7c2015b}
To perceive a complex means to perceive that its constituents are
related to one another in such and such a way. This no doubt also explains
why there are two possible ways of seeing the figure as a cube; and all
similar phenomena. For we really see two different facts. (If I look in the
first place at the corners marked a and only glance at the b's, then the
a's appear to be in front, and vice versa).
\end{itemize}
\end{itemize}
\subsubsection*{5.55}
\label{sec:orgefcd510}
We now have to answer a priori the question about all the possible
forms of elementary propositions. Elementary propositions consist of names.
Since, however, we are unable to give the number of names with different
meanings, we are also unable to give the composition of elementary
propositions.
\begin{itemize}
\item 5.551
\label{sec:org8b5efac}
Our fundamental principle is that whenever a question can be decided
by logic at all it must be possible to decide it without more ado. (And if
we get into a position where we have to look at the world for an answer to
such a problem, that shows that we are on a completely wrong track.)
\item 5.552
\label{sec:org2968d52}
The 'experience' that we need in order to understand logic is not
that something or other is the state of things, but that something is :
that, however, is not an experience. Logic is prior to every experience--
that something is so . It is prior to the question 'How?' not prior to the
question 'What?'
\begin{itemize}
\item 5.5521
\label{sec:org7950393}
And if this were not so, how could we apply logic? We might put it
in this way: if there would be a logic even if there were no world, how
then could there be a logic given that there is a world?
\end{itemize}
\item 5.553
\label{sec:org2fc4a9e}
Russell said that there were simple relations between different
numbers of things (individuals). But between what numbers? And how is this
supposed to be decided?--By experience? (There is no pre-eminent number.)
\item 5.554
\label{sec:org55f7487}
It would be completely arbitrary to give any specific form.
\begin{itemize}
\item 5.5541
\label{sec:org599bf37}
It is supposed to be possible to answer a priori the question
whether I can get into a position in which I need the sign for a 27-termed
relation in order to signify something.
\item 5.5542
\label{sec:orga4c5235}
But is it really legitimate even to ask such a question? Can we set
up a form of sign without knowing whether anything can correspond to it?
Does it make sense to ask what there must be in order that something can be
the case?
\end{itemize}
\item 5.555
\label{sec:org6871bc3}
Clearly we have some concept of elementary propositions quite apart
from their particular logical forms. But when there is a system by which we
can create symbols, the system is what is important for logic and not the
individual symbols. And anyway, is it really possible that in logic I
should have to deal with forms that I can invent? What I have to deal with
must be that which makes it possible for me to invent them.
\item 5.556
\label{sec:org5124622}
There cannot be a hierarchy of the forms of elementary propositions.
We can foresee only what we ourselves construct.
\begin{itemize}
\item 5.5561
\label{sec:org7afad84}
Empirical reality is limited by the totality of objects. The limit
also makes itself manifest in the totality of elementary propositions.
Hierarchies are and must be independent of reality.
\item 5.5562
\label{sec:orga1b5aa3}
If we know on purely logical grounds that there must be elementary
propositions, then everyone who understands propositions in their C form
must know It.
\item 5.5563
\label{sec:orgd869fd2}
In fact, all the propositions of our everyday language, just as they
stand, are in perfect logical order.--That utterly simple thing, which we
have to formulate here, is not a likeness of the truth, but the truth
itself in its entirety. (Our problems are not abstract, but perhaps the
most concrete that there are.)
\end{itemize}
\item 5.557
\label{sec:org0f87028}
The application of logic decides what elementary propositions there
are. What belongs to its application, logic cannot anticipate. It is clear
that logic must not clash with its application. But logic has to be in
contact with its application. Therefore logic and its application must not
overlap.
\begin{itemize}
\item 5.5571
\label{sec:org7c8cbbc}
If I cannot say a priori what elementary propositions there are,
then the attempt to do so must lead to obvious nonsense. 5.6 The limits of
my language mean the limits of my world.
\end{itemize}
\end{itemize}
\subsubsection*{5.61}
\label{sec:orgd2e32b7}
Logic pervades the world: the limits of the world are also its limits.
So we cannot say in logic, 'The world has this in it, and this, but not
that.' For that would appear to presuppose that we were excluding certain
possibilities, and this cannot be the case, since it would require that
logic should go beyond the limits of the world; for only in that way could
it view those limits from the other side as well. We cannot think what we
cannot think; so what we cannot think we cannot say either.
\subsubsection*{5.62}
\label{sec:orga2827f8}
This remark provides the key to the problem, how much truth there is
in solipsism. For what the solipsist means is quite correct; only it cannot
be said , but makes itself manifest. The world is my world: this is
manifest in the fact that the limits of language (of that language which
alone I understand) mean the limits of my world.
\begin{itemize}
\item 5.621
\label{sec:org8d1c0c6}
The world and life are one.
\end{itemize}
\subsubsection*{5.63}
\label{sec:org1965545}
I am my world. (The microcosm.)
\begin{itemize}
\item 5.631
\label{sec:orgeff2820}
There is no such thing as the subject that thinks or entertains
ideas. If I wrote a book called The World as l found it , I should have to
include a report on my body, and should have to say which parts were
subordinate to my will, and which were not, etc., this being a method of
isolating the subject, or rather of showing that in an important sense
there is no subject; for it alone could not be mentioned in that book.--
\item 5.632
\label{sec:org11fd29e}
The subject does not belong to the world: rather, it is a limit of
the world.
\item 5.633
\label{sec:org4f6b045}
Where in the world is a metaphysical subject to be found? You will
say that this is exactly like the case of the eye and the visual field. But
really you do not see the eye. And nothing in the visual field allows you
to infer that it is seen by an eye.
\begin{itemize}
\item 5.6331
\label{sec:org9dd79bd}
For the form of the visual field is surely not like this
\end{itemize}
\item 5.634
\label{sec:orgab21175}
This is connected with the fact that no part of our experience is at
the same time a priori. Whatever we see could be other than it is. Whatever
we can describe at all could be other than it is. There is no a priori
order of things.
\end{itemize}
\subsubsection*{5.64}
\label{sec:orgdb66211}
Here it can be seen that solipsism, when its implications are followed
out strictly, coincides with pure realism. The self of solipsism shrinks to
a point without extension, and there remains the reality co-ordinated with
it.
\begin{itemize}
\item 5.641
\label{sec:orgb1451a5}
Thus there really is a sense in which philosophy can talk about the
self in a non-psychological way. What brings the self into philosophy is
the fact that 'the world is my world'. The philosophical self is not the
human being, not the human body, or the human soul, with which psychology
deals, but rather the metaphysical subject, the limit of the world--not a
part of it.
\end{itemize}
\section*{6}
\label{sec:orge701864}
The general form of a truth-function is [p, E, N(E)]. This is the general
form of a proposition.
\begin{itemize}
\item 6.001
\label{sec:orgf493e81}
What this says is just that every proposition is a result of
successive applications to elementary propositions of the operation N(E)
\item 6.002
\label{sec:orgdf919cc}
If we are given the general form according to which propositions are
constructed, then with it we are also given the general form according to
which one proposition can be generated out of another by means of an
operation.
\end{itemize}
\subsubsection*{6.01}
\label{sec:org9a2652b}
Therefore the general form of an operation /'(n) is [E, N(E)] ' (n) (
= [n, E, N(E)]). This is the most general form of transition from one
proposition to another.
\subsubsection*{6.02}
\label{sec:org99169f6}
And this is how we arrive at numbers. I give the following definitions
x = \emph{0x Def., /'/v'x = /v+1'x Def. So, in accordance with these rules,
which deal with signs, we write the series x, /'x, /'}'x, \emph{'}'/'x, \ldots{} , in
the following way /0'x, /0+1'x, /0+1+1'x, /0+1+1+1'x, \ldots{} . Therefore,
instead of '[x, E, /'E]', I write '[/0'x, /v'x, /v+1'x]'. And I give the
following definitions 0 + 1 = 1 Def., 0 + 1 + 1 = 2 Def., 0 + 1 + 1 +1 = 3
Def., (and so on).
\begin{itemize}
\item 6.021
\label{sec:org1e5a35c}
A number is the exponent of an operation.
\item 6.022
\label{sec:org5200b9b}
The concept of number is simply what is common to all numbers, the
general form of a number. The concept of number is the variable number. And
the concept of numerical equality is the general form of all particular
cases of numerical equality.
\end{itemize}
\subsubsection*{6.03}
\label{sec:org64d8422}
The general form of an integer is [0, E, E +1].
\begin{itemize}
\item 6.031
\label{sec:orgc70df47}
The theory of classes is completely superfluous in mathematics. This
is connected with the fact that the generality required in mathematics is
not accidental generality.
\end{itemize}
\subsection*{6.1}
\label{sec:org25253ff}
The propositions of logic are tautologies.
\subsubsection*{6.11}
\label{sec:org613f9cb}
Therefore the propositions of logic say nothing. (They are the
analytic propositions.)
\begin{itemize}
\item 6.111
\label{sec:org1727611}
All theories that make a proposition of logic appear to have content
are false. One might think, for example, that the words 'true' and 'false'
signified two properties among other properties, and then it would seem to
be a remarkable fact that every proposition possessed one of these
properties. On this theory it seems to be anything but obvious, just as,
for instance, the proposition, 'All roses are either yellow or red', would
not sound obvious even if it were true. Indeed, the logical proposition
acquires all the characteristics of a proposition of natural science and
this is the sure sign that it has been construed wrongly.
\item 6.112
\label{sec:org4f3a8ba}
The correct explanation of the propositions of logic must assign to
them a unique status among all propositions.
\item 6.113
\label{sec:org9a81f3c}
It is the peculiar mark of logical propositions that one can
recognize that they are true from the symbol alone, and this fact contains
in itself the whole philosophy of logic. And so too it is a very important
fact that the truth or falsity of non-logical propositions cannot be
recognized from the propositions alone.
\end{itemize}
\subsubsection*{6.12}
\label{sec:org79bd371}
The fact that the propositions of logic are tautologies shows the
formal--logical--properties of language and the world. The fact that a
tautology is yielded by this particular way of connecting its constituents
characterizes the logic of its constituents. If propositions are to yield a
tautology when they are connected in a certain way, they must have certain
structural properties. So their yielding a tautology when combined in this
shows that they possess these structural properties.
\begin{itemize}
\item 6.1201
\label{sec:orga816787}
For example, the fact that the propositions 'p' and 'Pp' in the
combination '(p . Pp)' yield a tautology shows that they contradict one
another. The fact that the propositions 'p z q', 'p', and 'q', combined
with one another in the form '(p z q) . (p) :z: (q)', yield a tautology
shows that q follows from p and p z q. The fact that '(x) . fxx :z: fa' is
a tautology shows that fa follows from (x) . fx. Etc. etc.
\item 6.1202
\label{sec:org70772a9}
It is clear that one could achieve the same purpose by using
contradictions instead of tautologies.
\item 6.1203
\label{sec:org1d39e14}
In order to recognize an expression as a tautology, in cases where
no generality-sign occurs in it, one can employ the following intuitive
method: instead of 'p', 'q', 'r', etc. I write 'TpF', 'TqF', 'TrF', etc.
Truth-combinations I express by means of brackets, e.g. and I use lines to
express the correlation of the truth or falsity of the whole proposition
with the truth-combinations of its truth-arguments, in the following way So
this sign, for instance, would represent the proposition p z q. Now, by way
of example, I wish to examine the proposition P(p .Pp) (the law of
contradiction) in order to determine whether it is a tautology. In our
notation the form 'PE' is written as and the form 'E . n' as Hence the
proposition P(p . Pp). reads as follows If we here substitute 'p' for 'q'
and examine how the outermost T and F are connected with the innermost
ones, the result will be that the truth of the whole proposition is
correlated with all the truth-combinations of its argument, and its falsity
with none of the truth-combinations.
\end{itemize}
\item 6.121
\label{sec:org130ef06}
The propositions of logic demonstrate the logical properties of
propositions by combining them so as to form propositions that say nothing.
This method could also be called a zero-method. In a logical proposition,
propositions are brought into equilibrium with one another, and the state
of equilibrium then indicates what the logical constitution of these
propositions must be.
\item 6.122
\label{sec:org8db2aeb}
It follows from this that we can actually do without logical
propositions; for in a suitable notation we can in fact recognize the
formal properties of propositions by mere inspection of the propositions
themselves.
\begin{itemize}
\item 6.1221
\label{sec:org8a07dc2}
If, for example, two propositions 'p' and 'q' in the combination 'p
z q' yield a tautology, then it is clear that q follows from p. For
example, we see from the two propositions themselves that 'q' follows from
'p z q . p', but it is also possible to show it in this way: we combine
them to form 'p z q . p :z: q', and then show that this is a tautology.
\item 6.1222
\label{sec:org076062b}
This throws some light on the question why logical propositions
cannot be confirmed by experience any more than they can be refuted by it.
Not only must a proposition of logic be irrefutable by any possible
experience, but it must also be unconfirmable by any possible experience.
\item 6.1223
\label{sec:org31f847e}
Now it becomes clear why people have often felt as if it were for us
to 'postulate ' the 'truths of logic'. The reason is that we can postulate
them in so far as we can postulate an adequate notation.
\item 6.1224
\label{sec:orge7cb3cb}
It also becomes clear now why logic was called the theory of forms
and of inference.
\end{itemize}
\item 6.123
\label{sec:orge92c0de}
Clearly the laws of logic cannot in their turn be subject to laws of
logic. (There is not, as Russell thought, a special law of contradiction
for each 'type'; one law is enough, since it is not applied to itself.)
\begin{itemize}
\item 6.1231
\label{sec:orga75b9e8}
The mark of a logical proposition is not general validity. To be
general means no more than to be accidentally valid for all things. An
ungeneralized proposition can be tautological just as well as a generalized
one.
\item 6.1232
\label{sec:org6fdda24}
The general validity of logic might be called essential, in contrast
with the accidental general validity of such propositions as 'All men are
mortal'. Propositions like Russell's 'axiom of reducibility' are not
logical propositions, and this explains our feeling that, even if they were
true, their truth could only be the result of a fortunate accident.
\item 6.1233
\label{sec:org92367de}
It is possible to imagine a world in which the axiom of reducibility
is not valid. It is clear, however, that logic has nothing to do with the
question whether our world really is like that or not.
\end{itemize}
\item 6.124
\label{sec:orga2e41d5}
The propositions of logic describe the scaffolding of the world, or
rather they represent it. They have no 'subject-matter'. They presuppose
that names have meaning and elementary propositions sense; and that is
their connexion with the world. It is clear that something about the world
must be indicated by the fact that certain combinations of symbols--whose
essence involves the possession of a determinate character--are
tautologies. This contains the decisive point. We have said that some
things are arbitrary in the symbols that we use and that some things are
not. In logic it is only the latter that express: but that means that logic
is not a field in which we express what we wish with the help of signs, but
rather one in which the nature of the absolutely necessary signs speaks for
itself. If we know the logical syntax of any sign-language, then we have
already been given all the propositions of logic.
\item 6.125
\label{sec:org6d706b9}
It is possible--indeed possible even according to the old conception
of logic--to give in advance a description of all 'true' logical
propositions.
\begin{itemize}
\item 6.1251
\label{sec:org18a625a}
Hence there can never be surprises in logic.
\end{itemize}
\item 6.126
\label{sec:orgc83fd0c}
One can calculate whether a proposition belongs to logic, by
calculating the logical properties of the symbol. And this is what we do
when we 'prove' a logical proposition. For, without bothering about sense
or meaning, we construct the logical proposition out of others using only
rules that deal with signs . The proof of logical propositions consists in
the following process: we produce them out of other logical propositions by
successively applying certain operations that always generate further
tautologies out of the initial ones. (And in fact only tautologies follow
from a tautology.) Of course this way of showing that the propositions of
logic are tautologies is not at all essential to logic, if only because the
propositions from which the proof starts must show without any proof that
they are tautologies.
\begin{itemize}
\item 6.1261
\label{sec:org56c55ff}
In logic process and result are equivalent. (Hence the absence of
surprise.)
\item 6.1262
\label{sec:org0f4b95a}
Proof in logic is merely a mechanical expedient to facilitate the
recognition of tautologies in complicated cases.
\item 6.1263
\label{sec:orgd214bb1}
Indeed, it would be altogether too remarkable if a proposition that
had sense could be proved logically from others, and so too could a logical
proposition. It is clear from the start that a logical proof of a
proposition that has sense and a proof in logic must be two entirely
different things.
\item 6.1264
\label{sec:org04a4c57}
A proposition that has sense states something, which is shown by its
proof to be so. In logic every proposition is the form of a proof. Every
proposition of logic is a modus ponens represented in signs. (And one
cannot express the modus ponens by means of a proposition.)
\item 6.1265
\label{sec:orgf13c833}
It is always possible to construe logic in such a way that every
proposition is its own proof.
\end{itemize}
\item 6.127
\label{sec:org0c9cba4}
All the propositions of logic are of equal status: it is not the case
that some of them are essentially derived propositions. Every tautology
itself shows that it is a tautology.
\begin{itemize}
\item 6.1271
\label{sec:org1726345}
It is clear that the number of the 'primitive propositions of logic'
is arbitrary, since one could derive logic from a single primitive
proposition, e.g. by simply constructing the logical product of Frege's
primitive propositions. (Frege would perhaps say that we should then no
longer have an immediately self-evident primitive proposition. But it is
remarkable that a thinker as rigorous as Frege appealed to the degree of
self-evidence as the criterion of a logical proposition.)
\end{itemize}
\end{itemize}
\subsubsection*{6.13}
\label{sec:org35345f6}
Logic is not a body of doctrine, but a mirror-image of the world.
Logic is transcendental.
\subsection*{6.2}
\label{sec:orgbb3ade3}
Mathematics is a logical method. The propositions of mathematics are
equations, and therefore pseudo-propositions.
\subsubsection*{6.21}
\label{sec:org88dbce4}
A proposition of mathematics does not express a thought.
\begin{itemize}
\item 6.211
\label{sec:orgf651516}
Indeed in real life a mathematical proposition is never what we want.
Rather, we make use of mathematical propositions only in inferences from
propositions that do not belong to mathematics to others that likewise do
not belong to mathematics. (In philosophy the question, 'What do we
actually use this word or this proposition for?' repeatedly leads to
valuable insights.)
\end{itemize}
\subsubsection*{6.22}
\label{sec:org6b87d7d}
The logic of the world, which is shown in tautologies by the
propositions of logic, is shown in equations by mathematics.
\subsubsection*{6.23}
\label{sec:org79da6d4}
If two expressions are combined by means of the sign of equality, that
means that they can be substituted for one another. But it must be manifest
in the two expressions themselves whether this is the case or not. When two
expressions can be substituted for one another, that characterizes their
logical form.
\begin{itemize}
\item 6.231
\label{sec:orgc991bdf}
It is a property of affirmation that it can be construed as double
negation. It is a property of '1 + 1 + 1 + 1' that it can be construed as
'(1 + 1) + (1 + 1)'.
\item 6.232
\label{sec:org59e9040}
Frege says that the two expressions have the same meaning but
different senses. But the essential point about an equation is that it is
not necessary in order to show that the two expressions connected by the
sign of equality have the same meaning, since this can be seen from the two
expressions themselves.
\begin{itemize}
\item 6.2321
\label{sec:org405dcac}
And the possibility of proving the propositions of mathematics means
simply that their correctness can be perceived without its being necessary
that what they express should itself be compared with the facts in order to
determine its correctness.
\item 6.2322
\label{sec:org209a29b}
It is impossible to assert the identity of meaning of two
expressions. For in order to be able to assert anything about their
meaning, I must know their meaning, and I cannot know their meaning without
knowing whether what they mean is the same or different.
\item 6.2323
\label{sec:orgea14df4}
An equation merely marks the point of view from which I consider the
two expressions: it marks their equivalence in meaning.
\end{itemize}
\item 6.233
\label{sec:org7c14b62}
The question whether intuition is needed for the solution of
mathematical problems must be given the answer that in this case language
itself provides the necessary intuition.
\begin{itemize}
\item 6.2331
\label{sec:org0ff268d}
The process of calculating serves to bring about that intuition.
Calculation is not an experiment.
\end{itemize}
\item 6.234
\label{sec:org42462ec}
Mathematics is a method of logic.
\begin{itemize}
\item 6.2341
\label{sec:orgf86adc7}
It is the essential characteristic of mathematical method that it
employs equations. For it is because of this method that every proposition
of mathematics must go without saying.
\end{itemize}
\end{itemize}
\subsubsection*{6.24}
\label{sec:org1effe23}
The method by which mathematics arrives at its equations is the method
of substitution. For equations express the substitutability of two
expressions and, starting from a number of equations, we advance to new
equations by substituting different expressions in accordance with the
equations.
\begin{itemize}
\item 6.241
\label{sec:org524fe27}
Thus the proof of the proposition 2 t 2 = 4 runs as follows: (\emph{v)n'x
= /v x u'x Def., /2 x 2'x = (/2)2'x = (/2)1 + 1'x = /2' /2'x = /1 + 1'/1 +
1'x = (}'\emph{)'(}'\emph{)'x =}'\emph{'}'/'x = /1 + 1 + 1 + 1'x = /4'x. 6.3 The
exploration of logic means the exploration of everything that is subject to
law . And outside logic everything is accidental.
\end{itemize}
\subsubsection*{6.31}
\label{sec:org4d4b0ba}
The so-called law of induction cannot possibly be a law of logic,
since it is obviously a proposition with sense.---Nor, therefore, can it be
an a priori law.
\subsubsection*{6.32}
\label{sec:orgec39d15}
The law of causality is not a law but the form of a law.
\begin{itemize}
\item 6.321
\label{sec:org9e98db4}
'Law of causality'--that is a general name. And just as in mechanics,
for example, there are 'minimum-principles', such as the law of least
action, so too in physics there are causal laws, laws of the causal form.
\begin{itemize}
\item 6.3211
\label{sec:orga3985e6}
Indeed people even surmised that there must be a 'law of least
action' before they knew exactly how it went. (Here, as always, what is
certain a priori proves to be something purely logical.)
\end{itemize}
\end{itemize}
\subsubsection*{6.33}
\label{sec:org1f32db1}
We do not have an a priori belief in a law of conservation, but rather
a priori knowledge of the possibility of a logical form.
\subsubsection*{6.34}
\label{sec:org5f01576}
All such propositions, including the principle of sufficient reason,
tile laws of continuity in nature and of least effort in nature, etc. etc.--
all these are a priori insights about the forms in which the propositions
of science can be cast.
\begin{itemize}
\item 6.341
\label{sec:org4d0af76}
Newtonian mechanics, for example, imposes a unified form on the
description of the world. Let us imagine a white surface with irregular
black spots on it. We then say that whatever kind of picture these make, I
can always approximate as closely as I wish to the description of it by
covering the surface with a sufficiently fine square mesh, and then saying
of every square whether it is black or white. In this way I shall have
imposed a unified form on the description of the surface. The form is
optional, since I could have achieved the same result by using a net with a
triangular or hexagonal mesh. Possibly the use of a triangular mesh would
have made the description simpler: that is to say, it might be that we
could describe the surface more accurately with a coarse triangular mesh
than with a fine square mesh (or conversely), and so on. The different nets
correspond to different systems for describing the world. Mechanics
determines one form of description of the world by saying that all
propositions used in the description of the world must be obtained in a
given way from a given set of propositions--the axioms of mechanics. It
thus supplies the bricks for building the edifice of science, and it says,
'Any building that you want to erect, whatever it may be, must somehow be
constructed with these bricks, and with these alone.' (Just as with the
number-system we must be able to write down any number we wish, so with the
system of mechanics we must be able to write down any proposition of
physics that we wish.)
\item 6.342
\label{sec:org6fa9d15}
And now we can see the relative position of logic and mechanics. (The
net might also consist of more than one kind of mesh: e.g. we could use
both triangles and hexagons.) The possibility of describing a picture like
the one mentioned above with a net of a given form tells us nothing about
the picture. (For that is true of all such pictures.) But what does
characterize the picture is that it can be described completely by a
particular net with a particular size of mesh. Similarly the possibility of
describing the world by means of Newtonian mechanics tells us nothing about
the world: but what does tell us something about it is the precise way in
which it is possible to describe it by these means. We are also told
something about the world by the fact that it can be described more simply
with one system of mechanics than with another.
\item 6.343
\label{sec:org3a92385}
Mechanics is an attempt to construct according to a single plan all
the true propositions that we need for the description of the world.
\begin{itemize}
\item 6.3431
\label{sec:org6e19f3f}
The laws of physics, with all their logical apparatus, still speak,
however indirectly, about the objects of the world.
\item 6.3432
\label{sec:orgbfda09b}
We ought not to forget that any description of the world by means of
mechanics will be of the completely general kind. For example, it will
never mention particular point-masses: it will only talk about any point-
masses whatsoever.
\end{itemize}
\end{itemize}
\subsubsection*{6.35}
\label{sec:org5078e97}
Although the spots in our picture are geometrical figures,
nevertheless geometry can obviously say nothing at all about their actual
form and position. The network, however, is purely geometrical; all its
properties can be given a priori. Laws like the principle of sufficient
reason, etc. are about the net and not about what the net describes.
\subsubsection*{6.36}
\label{sec:orgf13a001}
If there were a law of causality, it might be put in the following
way: There are laws of nature. But of course that cannot be said: it makes
itself manifest.
\begin{itemize}
\item 6.361
\label{sec:org7070508}
One might say, using Hertt:'s terminology, that only connexions that
are subject to law are thinkable.
\begin{itemize}
\item 6.3611
\label{sec:orgf208f20}
We cannot compare a process with 'the passage of time'--there is no
such thing--but only with another process (such as the working of a
chronometer). Hence we can describe the lapse of time only by relying on
some other process. Something exactly analogous applies to space: e.g. when
people say that neither of two events (which exclude one another) can
occur, because there is nothing to cause the one to occur rather than the
other, it is really a matter of our being unable to describe one of the two
events unless there is some sort of asymmetry to be found. And if such an
asymmetry is to be found, we can regard it as the cause of the occurrence
of the one and the non-occurrence of the other.
\begin{itemize}
\item 6.36111
\label{sec:orgfad6c91}
Kant's problem about the right hand and the left hand, which cannot
be made to coincide, exists even in two dimensions. Indeed, it exists in
one-dimensional space in which the two congruent figures, a and b, cannot
be made to coincide unless they are moved out of this space. The right hand
and the left hand are in fact completely congruent. It is quite irrelevant
that they cannot be made to coincide. A right-hand glove could be put on
the left hand, if it could be turned round in four-dimensional space.
\end{itemize}
\end{itemize}
\item 6.362
\label{sec:org8e3ea92}
What can be described can happen too: and what the law of causality
is meant to exclude cannot even be described.
\item 6.363
\label{sec:orga4ea3e0}
The procedure of induction consists in accepting as true the simplest
law that can be reconciled with our experiences.
\begin{itemize}
\item 6.3631
\label{sec:org20f5f28}
This procedure, however, has no logical justification but only a
psychological one. It is clear that there are no grounds for believing that
the simplest eventuality will in fact be realized.
\begin{itemize}
\item 6.36311
\label{sec:orga59c670}
It is an hypothesis that the sun will rise tomorrow: and this means
that we do not know whether it will rise.
\end{itemize}
\end{itemize}
\end{itemize}
\subsubsection*{6.37}
\label{sec:org425280b}
There is no compulsion making one thing happen because another has
happened. The only necessity that exists is logical necessity.
\begin{itemize}
\item 6.371
\label{sec:org19f019d}
The whole modern conception of the world is founded on the illusion
that the so-called laws of nature are the explanations of natural
phenomena.
\item 6.372
\label{sec:orgf755ae8}
Thus people today stop at the laws of nature, treating them as
something inviolable, just as God and Fate were treated in past ages. And
in fact both are right and both wrong: though the view of the ancients is
clearer in so far as they have a clear and acknowledged terminus, while the
modern system tries to make it look as if everything were explained.
\item 6.373
\label{sec:org213a8f6}
The world is independent of my will.
\item 6.374
\label{sec:org1f7994c}
Even if all that we wish for were to happen, still this would only be
a favour granted by fate, so to speak: for there is no logical connexion
between the will and the world, which would guarantee it, and the supposed
physical connexion itself is surely not something that we could will.
\item 6.375
\label{sec:org2c9b6a3}
Just as the only necessity that exists is logical necessity, so too
the only impossibility that exists is logical impossibility.
\begin{itemize}
\item 6.3751
\label{sec:orgeba93fe}
For example, the simultaneous presence of two colours at the same
place in the visual field is impossible, in fact logically impossible,
since it is ruled out by the logical structure of colour. Let us think how
this contradiction appears in physics: more or less as follows--a particle
cannot have two velocities at the same time; that is to say, it cannot be
in two places at the same time; that is to say, particles that are in
different places at the same time cannot be identical. (It is clear that
the logical product of two elementary propositions can neither be a
tautology nor a contradiction. The statement that a point in the visual
field has two different colours at the same time is a contradiction.)
\end{itemize}
\end{itemize}
\subsection*{6.4}
\label{sec:orge508478}
All propositions are of equal value.
\subsubsection*{6.41}
\label{sec:orgadb2f4f}
The sense of the world must lie outside the world. In the world
everything is as it is, and everything happens as it does happen: in it no
value exists--and if it did exist, it would have no value. If there is any
value that does have value, it must lie outside the whole sphere of what
happens and is the case. For all that happens and is the case is
accidental. What makes it non-accidental cannot lie within the world, since
if it did it would itself be accidental. It must lie outside the world.
\subsubsection*{6.42}
\label{sec:org082bdd6}
So too it is impossible for there to be propositions of ethics.
Propositions can express nothing that is higher.
\begin{itemize}
\item 6.421
\label{sec:orgef0ed57}
It is clear that ethics cannot be put into words. Ethics is
transcendental. (Ethics and aesthetics are one and the same.)
\item 6.422
\label{sec:org5f802c6}
When an ethical law of the form, 'Thou shalt \ldots{}' is laid down, one's
first thought is, 'And what if I do, not do it?' It is clear, however, that
ethics has nothing to do with punishment and reward in the usual sense of
the terms. So our question about the consequences of an action must be
unimportant.--At least those consequences should not be events. For there
must be something right about the question we posed. There must indeed be
some kind of ethical reward and ethical punishment, but they must reside in
the action itself. (And it is also clear that the reward must be something
pleasant and the punishment something unpleasant.)
\item 6.423
\label{sec:orgd9888a5}
It is impossible to speak about the will in so far as it is the
subject of ethical attributes. And the will as a phenomenon is of interest
only to psychology.
\end{itemize}
\subsubsection*{6.43}
\label{sec:org4d69e03}
If the good or bad exercise of the will does alter the world, it can
alter only the limits of the world, not the facts--not what can be
expressed by means of language. In short the effect must be that it becomes
an altogether different world. It must, so to speak, wax and wane as a
whole. The world of the happy man is a different one from that of the
unhappy man.
\begin{itemize}
\item 6.431
\label{sec:org35bb7df}
So too at death the world does not alter, but comes to an end.
\begin{itemize}
\item 6.4311
\label{sec:org8a31031}
Death is not an event in life: we do not live to experience death.
If we take eternity to mean not infinite temporal duration but
timelessness, then eternal life belongs to those who live in the present.
Our life has no end in just the way in which our visual field has no
limits.
\item 6.4312
\label{sec:orgaf98341}
Not only is there no guarantee of the temporal immortality of the
human soul, that is to say of its eternal survival after death; but, in any
case, this assumption completely fails to accomplish the purpose for which
it has always been intended. Or is some riddle solved by my surviving for
ever? Is not this eternal life itself as much of a riddle as our present
life? The solution of the riddle of life in space and time lies outside
space and time. (It is certainly not the solution of any problems of
natural science that is required.)
\end{itemize}
\item 6.432
\label{sec:orgf3c03e2}
How things are in the world is a matter of complete indifference for
what is higher. God does not reveal himself in the world.
\begin{itemize}
\item 6.4321
\label{sec:org2f45f3e}
The facts all contribute only to setting the problem, not to its
solution.
\end{itemize}
\end{itemize}
\subsubsection*{6.44}
\label{sec:orgf8dd843}
It is not how things are in the world that is mystical, but that it
exists.
\subsubsection*{6.45}
\label{sec:orge0630ea}
To view the world sub specie aeterni is to view it as a whole--a
limited whole. Feeling the world as a limited whole--it is this that is
mystical.
\subsection*{6.5}
\label{sec:org35493cb}
When the answer cannot be put into words, neither can the question be
put into words. The riddle does not exist. If a question can be framed at
all, it is also possible to answer it.
\subsubsection*{6.51}
\label{sec:orga311633}
Scepticism is not irrefutable, but obviously nonsensical, when it
tries to raise doubts where no questions can be asked. For doubt can exist
only where a question exists, a question only where an answer exists, and
an answer only where something can be said.
\subsubsection*{6.52}
\label{sec:orgdf4ddaa}
We feel that even when all possible scientific questions have been
answered, the problems of life remain completely untouched. Of course there
are then no questions left, and this itself is the answer.
\begin{itemize}
\item 6.521
\label{sec:org1f577be}
The solution of the problem of life is seen in the vanishing of the
problem. (Is not this the reason why those who have found after a long
period of doubt that the sense of life became clear to them have then been
unable to say what constituted that sense?)
\item 6.522
\label{sec:org1f4efd2}
There are, indeed, things that cannot be put into words. They make
themselves manifest. They are what is mystical.
\end{itemize}
\subsubsection*{6.53}
\label{sec:orgcda47a9}
The correct method in philosophy would really be the following: to say
nothing except what can be said, i.e. propositions of natural science--i.e.
something that has nothing to do with philosophy -- and then, whenever
someone else wanted to say something metaphysical, to demonstrate to him
that he had failed to give a meaning to certain signs in his propositions.
Although it would not be satisfying to the other person--he would not have
the feeling that we were teaching him philosophy--this method would be the
only strictly correct one.
\subsubsection*{6.54}
\label{sec:org595923a}
My propositions are elucidatory in this way: he who understands me
finally recognizes them as senseless, when he has climbed out through them,
on them, over them. (He must so to speak throw away the ladder, after he
has climbed up on it.) He must transcend these propositions, and then he
will see the world aright.
\section*{7}
\label{sec:org0651e53}
What we cannot speak about we must pass over in silence.
\end{document}
